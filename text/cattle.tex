\documentclass[12pt]{article}
\usepackage{amsmath,amsthm}
\usepackage[utf8]{inputenc}
\usepackage[T1]{fontenc}
%\usepackage{minionpro}
\usepackage{array,graphicx}

\setlength{\topmargin}{-0.3in} \setlength{\textheight}{8.75in}
\setlength{\oddsidemargin}{0.0in} \setlength{\evensidemargin}{0.0in}
\setlength{\textwidth}{6.5in}
\def\labelenumi{\arabic{enumi}.}
\def\theenumi{\arabic{enumi}}
\def\labelenumii{(\alph{enumii})}
\def\theenumii{\alph{enumii}}
\def\p@enumii{\theenumi.}
\def\labelenumiii{\arabic{enumiii}.}
\def\theenumiii{\arabic{enumiii}}
\def\p@enumiii{(\theenumi)(\theenumii)}
\def\labelenumiv{\arabic{enumiv}.}
\def\theenumiv{\arabic{enumiv}}
\def\p@enumiv{\p@enumiii.\theenumiii}
\pagestyle{plain}
\pagestyle{plain} \setcounter{secnumdepth}{3}
\newcommand{\D}{\mathop{\mathrm{d\mathstrut}}\nolimits\!}
\newcommand{\dt}{\D t}
\newcommand{\dz}{\D z}
\newcommand{\E}{\mathop{\mathrm{E\mathstrut}}\nolimits}
\newcommand{\Var}{\mathop{\mathrm{Var\mathstrut}}\nolimits}
\newcommand{\sd}{\mathop{\mathrm{sd\mathstrut}}\nolimits}
\newcommand{\diag}{\mathop{\mathrm{diag\mathstrut}}\nolimits}
\newcommand{\Cov}{\mathop{\mathrm{Cov\mathstrut}}\nolimits}
\newcommand{\Corr}{\mathop{\mathrm{Corr\mathstrut}}\nolimits}
\newtheorem{definition}{Definition}
\newtheorem{proposition}{Proposition}
\newtheorem{conjecture}{Conjecture}
\newtheorem{moment}{Empirical regularity}
\newtheorem{insight}{Qualitative prediction}

\newcommand{\dofigure}[2]{\begin{figure}
\begin{centering}
\includegraphics[width=0.75\linewidth]{figures/#1}
  \caption{#2\label{fig:#1}}
\end{centering}
\end{figure}}

\newcommand{\dotable}[2]{\begin{table}[h!]
\begin{centering}
\caption{#2\label{tab:#1}}
\includegraphics[width=0.75\linewidth]{figures/#1}
\end{centering}
\end{table}}

\begin{document}

\title{Cattle, Steaks and Restaurants: Development Accounting when Space Matters}
\author{Péter Karádi\thanks{European Central Bank. E-mail: peter.karadi@ecb.int}~ and Miklós Koren\thanks{Central European University, IE--HAS and CEPR. E-mail: korenm@ceu.hu.}}
\maketitle

\begin{abstract}
\end{abstract}

Productivity is much lower in poor countries than in rich ones (REFS). To understand the fundamental causes of productivity differences, it is important to identify the sectors in which these differences are greatest (REF). Several recent papers have studied the sectoral composition of productivity differences by using data on sector-level prices (REF). The main result is XXX. Intuitively, prices are XXX

We revisit the measurement of sectoral prices and sectoral productivity in a macro model where land and location play a role. A high price in one sector may simply reflect high rents accruing to a non-reproducible input, land. For example, the fact that restaurants are expensive in New York City has more to with high rents, and may not imply that NYC restaurants are inefficient. Rents are, in turn, determined in general equilibrium, and may respond to demand for land in other activities, such as finance, culture and housing in the NYC example.

We build a multi-sector general equilibrium model. Each sector uses labor (or a composite of other spatially mobile inputs) and land. The location of sectors is determined in the canonical von Thünen city model. XXX DESCRIBE MODEL

Our model yields a simple spatial equilibrium in which agriculture (``cattle'') locates farthest away from the center, manufacturing (``steaks'') occupies a ring outside the center, and services (``restaurants'') are in a central circle.

There are two reasons why development accounting in our model is different from models without land. First, as some sectors are more land intensive, their prices may be more sensitive to rents. Because of this, conditional on productivity, agricultural prices will be relatively higher in rich countries. Second, because sectors endogenously choose locations, their price is also affected by the rent gradient: the speed with which rents decline in distance from the city center. Urban sectors will be relatively more expensive in countries where the rent gradient is higher.

To quantify the importance of these two mechanisms, we calibrate our model to match the spatial distribution of economic activities in the U.S., and the share of sectors in each country.

\section{A model of }
\end{document}

