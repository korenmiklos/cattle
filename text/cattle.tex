\documentclass[12pt]{article}
\usepackage{amsmath,amsthm}
\usepackage[utf8]{inputenc}
%\usepackage[T1]{fontenc}
%\usepackage{minionpro}
\usepackage{array,graphicx}
\usepackage{chicago}        %bibliography style
\usepackage{booktabs}       %to make table lines thicker

\setlength{\topmargin}{-0.3in} \setlength{\textheight}{8.75in}
\setlength{\oddsidemargin}{0.25in} \setlength{\evensidemargin}{0.25in}
\setlength{\textwidth}{6in}
\def\labelenumi{\arabic{enumi}.}
\def\theenumi{\arabic{enumi}}
\def\labelenumii{(\alph{enumii})}
\def\theenumii{\alph{enumii}}
\def\p@enumii{\theenumi.}
\def\labelenumiii{\arabic{enumiii}.}
\def\theenumiii{\arabic{enumiii}}
\def\p@enumiii{(\theenumi)(\theenumii)}
\def\labelenumiv{\arabic{enumiv}.}
\def\theenumiv{\arabic{enumiv}}
\def\p@enumiv{\p@enumiii.\theenumiii}
\pagestyle{plain}
\pagestyle{plain} \setcounter{secnumdepth}{3}
\newcommand{\D}{\mathop{\mathrm{d\mathstrut}}\nolimits\!}
\newcommand{\dt}{\D t}
\newcommand{\dz}{\D z}
\newcommand{\E}{\mathop{\mathrm{E\mathstrut}}\nolimits}
\newcommand{\Var}{\mathop{\mathrm{Var\mathstrut}}\nolimits}
\newcommand{\sd}{\mathop{\mathrm{sd\mathstrut}}\nolimits}
\newcommand{\diag}{\mathop{\mathrm{diag\mathstrut}}\nolimits}
\newcommand{\Cov}{\mathop{\mathrm{Cov\mathstrut}}\nolimits}
\newcommand{\Corr}{\mathop{\mathrm{Corr\mathstrut}}\nolimits}
\newtheorem{definition}{Definition}
\newtheorem{proposition}{Proposition}
\newtheorem{conjecture}{Conjecture}
\newtheorem{moment}{Empirical regularity}
\newtheorem{insight}{Qualitative prediction}

\renewcommand{\baselinestretch}{1.17}

\newcommand{\dofigure}[3]{\begin{figure}
\begin{centering}
\includegraphics[width=0.75\linewidth]{figures/#1}
  \caption{#2\label{fig:#1}}
\end{centering}

\noindent \footnotesize{#3}
\end{figure}}

\newcommand{\dotable}[2]{\begin{table}[h!]
\begin{centering}
\caption{#2\label{tab:#1}}
\includegraphics[width=0.75\linewidth]{figures/#1}
\end{centering}
\end{table}}

\begin{document}

\title{Cattle, Steaks and Restaurants:\\ Development Accounting when Space Matters\thanks{For useful comments, we thank Thomas Holmes, Esteban Rossi-Hansberg, Albert Saiz, Adam Szeidl, Jonathan Vogel, Kei-Mu Yi and audiences of the Conference on Urban and Regional Economics and seminars at Central European University, Princeton University and the Philadelphia Fed.}}
\author{Péter Karádi\thanks{European Central Bank. E-mail: peter.karadi@ecb.int} and Miklós Koren\thanks{Central European University, MTA KRTK and CEPR. E-mail: korenm@ceu.edu.}}
\maketitle

\begin{abstract}
We conduct sector-level development accounting in a macro model where land and location play a role. Producers in agriculture, manufacturing and services choose their location to trade off transport costs to the city center and rents. We solve for the spatial equilibrium and show how space affects the aggregate production function and measured productivity. Studies not accounting for sector location will deem services in large, expensive cities unproductive. This biases development accounting because rich countries have large service-cities. Our preliminary calibrations show that, correcting for sector location, service productivity varies as much across countries as manufacturing productivity does. This is in contrast with previous studies that found smaller variation in service productivity.
\end{abstract}

Output per worker is much lower in poor countries than in rich ones (\citeN{Klenow97}, \citeN{Hall99}, \citeN{Caselli05}). To understand the fundamental causes of productivity differences, it is important to identify the sectors in which these differences are the greatest. Several recent papers have studied the sectoral composition of productivity differences by using data on sector-level inputs, outputs and prices (\citeN{Bailey01}, \citeN{Caselli05}, \citeN{Duarte10}, \citeN{Restuccia08}). Their main result is that productivity differences are sizeable and they are larger in agriculture than in manufacturing and services.\footnote{These results are in line with the classic Balassa-Samuelson literature (\citeN{Balassa64}, \citeN{Samuelson64}, \citeN{Baumol65}, \citeN{Baumol67}), which explains sectoral price-level differences by productivity advantage of manufacturing over services.} 

Labor productivity depends on the quantity or quality of inputs used in the production process. Here we argue that land input has been conspicously missing from development accounting exercises to date.Controlling for land when estimating macro productivity is important for two reasons. First, countries with high population density are relatively scarce in land and, to the extent that land matters in production, these countries will have low labor productivity. Second, land varies in a crucial quality component: its proximity to consumers. Urban land is more valuable because it is more productive in producing goods and services with high transport costs. Again, countries may differ in the availability of urban land. Smaller cities are associated with higher labor productivity. Accounting for land and location is especially important for sector-level development accounting. Agriculture uses land intensively and will be especially sensitive to controlling for the quantity of land. Services, in turn, locate in urban areas and will be especially sensitive to controlling for location. 

To understand how land and location affect development accounting, we build a simple multi-sector general equilibrium model. Each sector uses labor (or a composite of other spatially mobile inputs) and land. The location of sectors is determined as in the canonical von Thünen city model. In the model, all trade happens in the city center, the central business district (CBD). Producers choose their location freely on a plane, and have to pay a shipping cost to transport their goods to the CBD. This spatial structure introduces variable land-quality to the model, as land closer to the CBD saves on transport cost. Equilibrium rents decrease with distance from the CBD, and the producers optimally choose locations to balance savings on transport costs to higher rents. Our model yields a simple spatial equilibrium in which agriculture (``cattle'') locates farthest away from the center, manufacturing (``steaks'') occupies a ring outside the center, and services (``restaurants'') are in a central circle.

In our model, we can decompose output per worker into four components: (i) Hicks neutral productivity, (ii) capital and other sptially mobile imputs per worker, (iii) land per worker, and (iv) a term reflecting distance to consumers. Components (iii) and (iv) are novel to our model. First, as some sectors are more land intensive, their labor productivity will be more sensitive to the scarcity or abundance of land. Conditional on other productivity factors, agricultural output per worker will be higher in countries with an abundant supply of land. Second, urban sectors (such as services) will be relatively more productive in countries where cities are smaller and urban rents are lower.

To quantify the importance of these two mechanisms, we calibrate our model in a set of developed and emerging-market countries with comparable data on sector level productivites and prices.\footnote{We are using the EUKLEMS database, see \citeN{OMahony09}.} We set common technology parameters to match the sectoral land-shares and spatial distribution of economic activities in the U.S., and allow international variation in the level of urbanization and sector-shares of output. We then use the calibrated model to decompose output per worker into the four components for each country. 

XX RESULTS

\bigskip

Beyond the development accounting papers cited above, our work is related to MACRO URBAN


\section{Motivating facts}
Our model is motivated by four robust empirical facts.

First, urban rents, relative to rural rents increase with development. Urban land becomes more important both in absolute terms and relatively (Clark, 2007).

Second, the relative price of services increase with development. This is known as Baumol's cost disease or the Balassa-Samuelson effect.

Third, rich countries are more urbanized.

Fourth, rich countries produce less agriculture and more services.


\section{A model of industry location}
We introduce location choice to a multi-sector general equilibrium model. We have three sectors: agriculture, manufacturing and services, each using land and labor for production. Our spatial structure follows the von Thünen monocentric city model: producers choose a location on the plane and need to transport their goods to the central business district. %They differ in the land intensity of their production (significantly higher with agriculture) and their transport costs (significantly higher with services).

\subsection{Consumers}
To be consistent with Fact 4, we assume non-homothetic consumer preferences. We follow Fieler (2011) in assuming an isoelastic utility function with different elasticities for the three sectors.\footnote{We have estimated the preferences proposed by Herrendorf, Rogerson and Valentinyi on cross-country data, but this provided a poor fit for rich countries, where the role of neccesity spending is vanishingly small.}
\begin{equation}
\label{eq:Utility}
u = \sum_{i=a,m,s}\alpha_i^{1/\sigma_i} 
	\frac {\sigma_i}
			{\sigma_i-1}
	c_i^{1-1/\sigma_i}
\end{equation}
with $\alpha_i>0$ is the weight of sector $i$ and $\sigma_i$ governs the sector's price and income elasticity.

There is a mass $N$ of consumers whoe supply one unit of labor inelastically, and rent land to producers in a competitive market. They are \emph{absentee landlords}: the rents they collect are independent of the place of employment.

The budget constraint of a representative consumer is
\begin{equation}
\label{eq:BudCons}
\sum_i P_i c_i 
	= W
	+ \frac 1N \int_z R(z)\tilde{L}(z)dz,
\end{equation}
where $R(z)$ is the rent as a function of the distance to the center (see later), and $W$ is the wage. Because labor is freely mobile, wages do not depend on location.

The sectoral consumption expenditures satisfy
\begin{equation*}
{P_i c_i}
=
\alpha_i
\lambda^{ - \sigma_i}
{ P_i^{1-\sigma_i}}
\end{equation*}
where $\lambda$ is the Lagrange multiplier of the budget constraint, which is decreasing in income. For a given set of prices, the expenditure of good $i$ relative to good $j$ is increasing in income if $\sigma_i>\sigma_j$. 

The income elasticity of good $i$ is
\[
\frac 	{\partial \ln c_i}
		{\partial \ln y}
=
\frac 	{\sigma_i}
		{\sum_j x_j\sigma_j},
\]
where $x_j$ is the expenditure share of good $j$
\subsection{Producers}
\subsubsection{Technology}
Output in sector $i$ at location $z$ depends on employment $N$ and land $L$ used at that location,
\[
Q_i(z) = A_i L_i(z)^{\beta_i}N_i(z)^{1-\beta_i}.
\]
We take labor to be freely mobile within the country, land is in fixed $\tilde{L}(z)$ supply in each location. The names ``land'' and ``labor'' are for the sake of convenience, these two factors correspond to spatially fixed and mobile factors, respectively, and we will calibrate them accordingly.

Sectors differ in their land shares $\beta_i$ and Hicks neutral productivity shifter $A_i$.

All products are sold and consumed at a single location, the central business district. This is location $z=0$, so that $z$ indexes distance to the center.

\subsubsection{Shipping}
To ship a product to the center, one has to incur shipping costs. If a unit of product $i$ leaves location $z$, only
\[
e^{-\tau_i z}
\]
units arrive at the center. This is akin to the iceberg assumption of transport costs. Sectors also differ in the intensity of shipping costs $\tau_i$.

\subsubsection{Profits}
Profits in sector $i$ from production at $z$ is
\begin{equation}
\label{eq:profit}
\Pi_i(z)=P_ie^{-\tau_iz}Q_i(z)-WN_i(z)-R(z)L_i(z).
\end{equation}
A sector being active at location $z$ requires that their maximized profit $\max\Pi_i(z)\geq0.$

\subsection{Single-city Equilibrium}
We begin by characterizing the equilibrium of a single city with fixed amount of land and labor.

The competitive spatial equilibrium in a circular city $z\in Z$ is an equilibrium set of quantities $\{C_i, Q_i(z), L_i(z), N_i(z)\}_{i=1}^3$ and prices $\{P_i, R(z), W\}_{i=1}^3$ such that
\begin{enumerate}
    \item The consumer chooses $\{C_i\}_{i=1}^3$ to maximize utility (\ref{eq:Utility}) subject to its budget constraint (\ref{eq:BudCons}), taking prices as given.
    \item The producers choose technology $i$ and location $z$, and nonnegative quantities $\{Q_i(z), L_i(z), N_i(z)\}$ to maximize their profits (\ref{eq:profit}), taking prices as given.
    \item Sectoral goods market clear: $C_i=\int_{z\in Z} e^{-\tau_iz}Q_i(z)dz$ for $i=1,2,3$.
    \item Labor market clears: $N=\sum_i\int_{z\in Z} N_i(z)$.
    \item Land markets clear at every location: $\sum_iL_i(z)=\tilde L(z)$ for all $z\in Z$.
\end{enumerate}

\subsection{Spatial structure}
The equilibrium, as we show below, has a simple and intuitive structure. Sectors with higher transport cost intensity ($\tau_i$) and lower land share ($\beta_i$) locate closer to the center. Realistically, services locate in a circle around the center, manufacturing goods are produced on a ring around it, and agriculture inhibits the outer ring.

To see why it is the case, it is instructive to construct the sectoral bid rent curves $R_i(z)$. These are the maximum rent an active producer with technology $i$ would be willing to pay at location $z$. %In a competitive market, the sector with the highest bid rent curve is active at a location ($R(z)=\max_iR_i(z)$ for all $z$).
A profit maximizing producer is choosing its land $N_i(z)$ and labor demand $L_i(z)$ to equalize the value marginal product of land and labor to rents and wages, respectively. It is true for any rent function or wages it might face, so it is true for rents given by its own bid-rent curve $R_i(z)$, in particular.
\begin{align}
R_i(z) &=\beta_i P_ie^{-\tau_i z}A_i \left(\frac{N_i(z)}{L_i(z)}\right)^{1-\beta_i}\\
W &=(1-\beta_i) P_ie^{-\tau_i z}A_i \left(\frac{N_i(z)}{L_i(z)}\right)^{-\beta_i}
\end{align}

The sectoral labor-land ratio (employment density) can be expressed as
\begin{equation}
\label{eq:EmpDens}
\frac{N_i(z)}{L_i(z)} = \frac{1-\beta_i}{\beta_i}\frac{R_i(z)}{W}.
\end{equation}
Substituting this into the FOC for land, we can get an implicit expression for the sectoral bid rent curve:
\[
R_i(z) =\beta_i P_ie^{-\tau_i z}A_i \left(\frac{1-\beta_i}{\beta_i}\frac{R_i(z)}{W}\right)^{1-\beta_i},
\]
%\[
%R_i(z)^{\beta_i} =\beta_i^{\beta_i}(1-\beta_i)^{1-\beta_i} P_ie^{-\tau_i z}A_i W^{\beta_i-1}
%\]
from which
\begin{equation}
\label{eq:BidRent}
R_i(z) =\beta_i(1-\beta_i)^{1/\beta_i-1} (P_iA_i)^{1/\beta_i} W^{1-1/\beta_i} e^{-\frac{\tau_i}{\beta_i} z}.
\end{equation}
Equation (\ref{eq:BidRent}) pins down the gradient of the sectoral rent curve, which determines how fast the bid-rents decrease with the distance from the CBD. The gradient is an increasing function of the sectoral transport costs and a decreasing function of land shares $|\partial\log R_i(z)/\partial \log z|=\tau_i/\beta_i$. Intuitively, transport costs ($\tau_iz$) increase with distance, so producers offer lower rents for farther locations. If land were the only factor of production ($\beta_i=1$), only this direct effect would be present and the gradient would only depend on the transport cost intensity. With labor present, however, the producers can substitute labor for land, introducing an indirect effect on the gradient. As the equilibrium land-labor ratio shows (see equation \ref{eq:EmpDens}), the land share $\beta_i$ determines the strength of this substitutability: higher land share implies bid-rent curves decreasing with a slower rate.

We can also get the employment gradient by substituting the bid rent curve into the labor-land ratio,
\begin{equation}
\label{eq:EmpGrad}
\frac{N_i(z)}{L_i(z)} = (1-\beta_i)^{1/\beta_i} \left(\frac{P_iA_i}{W}\right)^{1/\beta_i} e^{-\frac{\tau_i}{\beta_i} z}.
\end{equation}
This is the relationship we use to calibrate the sectoral transport cost intensities.

\begin{proposition}
Assume that $\tau_s/\beta_s>\tau_m/\beta_m>\tau_a/\beta_a>0$.\footnote{It is straightforward to see that in the knife-edge case of equality, the equilibrium were not unique and not necessarily a simple partitioning.} There exists a unique competitive spatial equilibrium. The structure of this equilibrium is a simple partitioning, with locations $0<z_1,z_2<z_3$ such that services with the higher $\tau_i/\beta_i$ rate locate closest to the CBD in $\left[0,z_1\right]$, manufacturing locates in $\left(z_1,z_2\right]$, and agriculture locates in $\left(z_2,z_3\right]$.
\end{proposition}
\begin{proof}
Sketch of proof:

Existence and uniqueness: The proof shows that there is exactly one vector of prices $(P_1, P_2, P_3)$, such that $R_i(z)\geq R_{-i}(z)$ for $z\in[z_{i-1},z_i]$ for $i=1,2,3.$ with goods and labor markets clearing. [TO BE COMPLETED]

Structure: To see why the spatial competitive equilibrium generates a simple partition, it is instructive to look at figure \ref{fig:BidRent}. The graphical proof is based on the shape of the sectoral bid rent curves.  Under our assumptions, sectoral bid-rent curves are strictly decreasing with location ($z$). Let's normalize $W=1$. The equilibrium prices $P_i$ influence the position of the bid-rent curves, with higher prices implying higher bid-rents for each location. In equilibrium, a sector $i$ is active in location $z$ if $R_i(z)\geq R_{-i}(z)$, where $-i$ denotes the other sectors. We are to show that in equilibrium there are a vector of locations $(z_1, z_2)$ ($z_3$, the urban fringe is exogenously given), such that $0\leq z_i\leq z_3$ and $R_i(z)\geq R_{-i}(z)$ for $z\in[z_{i-1},z_i]$ for $i=1,2,3.$

The equilibrium requires that $R_1(z)$ crosses once with $R_2(z)$ at $z=z_1$. Similarly, $R_2(z)$ crosses once with $R_3(z)$ at $z=z_2$. The strictly decreasing bid-rent curves imply that they cross \emph{at most} once. If equilibrium prices were such that one of them did not cross with any other at all, that sector would not produce. As our Cobb-Douglas utility function implies positive demand for each good, this would contradict the equilibrium. $z_1<z_2$, otherwise sector 1 would not produce. $z_2<z_3$, otherwise sector 3 would not produce.
\end{proof}

\begin{figure}[h!]
\label{fig:BidRent}
\caption{Spatial equilibrium}
\begin{center}
\includegraphics[scale=0.6]{figures/fig_spatial_equilibrium}
\end{center}

\noindent \footnotesize{The figure plots the structure of a spatial competitive equilibrium. It shows equilibrium sectoral bid rent curves as a function of distance from the city center. A sector is active over an area where it is willing to overbid alternative sectors. Crossings of the bid-rent curves determine the borders of the sectors. Services are active in $[0,z_1]$, manufacturing in $\left(z_1,z_2\right]$, and services in $\left(z_2,z_3\right]$.}
\end{figure}


\subsection{Spatial arbitrage}
Competitive land markets ensure that each location goes to the highest bidder. The slope of the bid-rent gradient in sector $i$ is $\tau_i/\beta_i$. Suppose that this is the highest for services, lower for manufacturing, and lowest for agriculture.

Let $z_i$ denote the outer edge of the land use of sector $i$. At such locations, both sector $i$ and sector $i+1$ have the same reservation rent:
\begin{align}\label{eq:BorderRents}
R(z_i) &=\beta_i(1-\beta_i)^{1/\beta_i-1} (P_iA_i)^{1/\beta_i} W^{1-1/\beta_i} e^{-\frac{\tau_i}{\beta_i} z_i}\\
R(z_i) &=\beta_{i+1}(1-\beta_{i+1})^{1/\beta_{i+1}-1} (P_{i+1}A_{i+1})^{1/\beta_{i+1}} W^{1-1/\beta_{i+1}} e^{-\frac{\tau_{i+1}}{\beta_{i+1}} z_i}
\end{align}
Substitute in the overall amount of value added, $Y_i = P_iQ_i$,
\[
R_i(z) = R_i e^{-\frac{\tau_i}{\beta_i}(z-\tilde z_i)}.
\]
\[
R_i = \frac{\beta_i Y_i}{L_i}
\]
\[
R_i(z) = \frac{\beta_i Y_i}{L_i} e^{-\frac{\tau_i}{\beta_i}(z-\tilde z_i)}.
\]
At the location $z_{i-1}$ where sectors $i$ and $i-1$ meet,
\[
\frac{\beta_{i-1} Y_{i-1}}{L_{i-1}}
  e^{-\frac{\tau_{i-1}}{\beta_{i-1}}(z_{i-1}-\tilde z_{i-1})} =
\frac{\beta_i Y_i}{L_i}
  e^{-\frac{\tau_i}{\beta_i}(z_{i-1} -\tilde z_i)}.
\]
\subsection{Solving for the prices}

At the sector borders $z_i$, neighboring sectors have the same reservation rents. Having solved for these locations, equations \ref{eq:BorderRents} provide implicit equations for relative product prices $P_i$ for $i=s,m.$ The consumer budget constraint, given by equation \ref{eq:BudCons}, gives us the equation for the remaining product price $P_a$. This closes the model description.   

\subsection{Circular city}




\section{Development accounting}
\subsection{Aggregation across space}
Sectoral production, as we show in this section, is a function of a suitably chosen representative location. 

Sectoral output measured at the center is
\begin{equation*}
\tilde{Q}_i=\int_{z\in Z_i}e^{-\tau_i z}A_iL_i(z)^\beta_iN_i(z)^{1-\beta_i}dz,
\end{equation*}
where $Z_i$ is the set of locations where sector $i$ is active in equilibrium. 

Let the representative location
\begin{equation}
\label{eq:ReprLoc}
\tilde z_i = -
\frac{\beta_i}{\tau_i}
\ln\int_{z\in Z_i} \frac{L_i(z)}{L_i}e^{-\frac{\tau_i}{\beta_i} z}dz
\end{equation}
denote the average distance of sector $i$ to the center, where $L_i=\int_{z\in Z_i} L_i(z)dz$ is the total amount of land devoted to sector $i$. This definition ensures that the trade cost going to location $\tilde z_i$ equals the land-weighted average trade cost across all sectoral locations,
\[
e^{-\frac{\tau_i}{\beta_i} \tilde z_i} = \int_{z\in Z_i} \frac{L_i(z)}{L_i}e^{-\frac{\tau_i}{\beta_i} z}dz.
\]
As we will see below, $\tilde z_i$ is a sufficient statistic about sector location for aggregation purposes. %(XX THIS IS A TRICK OF THE MELITZ MODEL)

\begin{proposition}
Aggregate sectoral production function is of the form: 
\begin{equation}
\tilde Q_i =
A_iL_i^{\beta_i}N_i^{1-\beta_i}
 e^{-\tau_i\tilde z_i}.
\end{equation}
It depends on trade costs from the representative location ($\tilde{z}_i$) defined by equation \ref{eq:ReprLoc} and is a Cobb-Douglas aggregate of sectoral land ($L_i=\int_{Z_i}L_i(z)dz$) and labor ($N_i=\int_{Z_i}N_i(z)dz$) use.
\end{proposition}

To show that it is the case, let us express value added (measured at the center) at a location $z$ per unit of land:
\[
\frac{e^{-\tau_i z} Q_i(z)}{L_i(z)} = e^{-\tau_i z} A_i(z)\left(\frac{N_i(z)}{L_i(z)}\right)^{1-\beta_i} = (1-\beta_i)^{1/\beta_i-1}
A_i^{1/\beta_i}\left(\frac{W}{P_i}\right)^{1-1/\beta_i}
 e^{-\frac{\tau_i}{\beta_i} z},
\]
where we used equation \ref{eq:EmpDens} on employment density. From this, total supply of sector $i$ becomes 
\begin{equation}
\label{eq:OutputInterm}
\tilde{Q_i} = \int_{z\in Z_i}\frac{e^{-\tau_i z} Q_i(z)}{L_i(z)}L_i(z)dz=(1-\beta_i)^{1/\beta_i-1}
(A_i)^{1/\beta_i}\left(\frac{W}{P_i}\right)^{1-1/\beta_i} L_i e^{-\frac{\tau_i}{\beta_i} \tilde z_i}.
\end{equation} 
by the definition of the representative location of the sector in equation \ref{eq:ReprLoc}.

Overall employment in the sector,
\[
N_i = \int_{z\in Z_i}\frac{N_i(z)}{L_i(z)}L_i(z)dz= (1-\beta_i)^{1/\beta_i}
\left(\frac{P_iA_i}{W}\right)^{1/\beta_i} L_i e^{-\frac{\tau_i}{\beta_i} \tilde z_i},
\]
so that we can express $W/P_i$ as
%\[
%\left(\frac{W}{P_i}\right)^{1/\beta_i} = (1-\beta_i)^{1/\beta_i}
%N_i^{-1}A_i^{1/\beta_i}
%L_i e^{-\frac{\tau_i}{\beta_i} \tilde z_i}
%\]
\[
\frac{W}{P_i} = (1-\beta_i)
N_i^{-\beta_i}A_i L_i^{\beta_i}
 e^{-\tau_i\tilde z_i}
\]
%\[
%\left(\frac{W}{P_i}\right)^{1-1/\beta_i} = (1-\beta_i)^{1-1/\beta_i}
%A_i^{1-1/\beta_i}N_i^{1-\beta_i}L_i^{\beta_i-1}
%e^{\frac{1-\beta_i}{\beta_i}\tau_i \tilde z_i}
%\]
Substituting this result to equation \ref{eq:OutputInterm}, we obtain the aggregate production function. 

\subsection{Productivity measurement}
Productivity measurements disregarding land and location lead to biased results, which we analyse below using our model. We first show the bias for labor productivity, which ignores land altogether, then for total-factor productivity, which accounts for land, but ignores its location.

%Suppose physical output is measured as revenue at producer prices deflated by a common price index. We take the price index to be the consumer price (XX CHECK IF IT MATTERS).
%\[
%\tilde Q_i(z) = \frac{e^{-\tau_i z}P_iQ_i(Z)}{P_i}
%\]
%We can write an aggregate output as
%\[
%\tilde Q_i =
%(1-\beta_i)^{1/\beta_i-1}
%A_i^{1/\beta_i}\left(\frac{W}{P_i}\right)^{1-1/\beta_i}
%L_i e^{-\frac{\tau_i}{\beta_i} \tilde z_i}
%\]
%which takes us to the aggregate production function
%\[
%\tilde Q_i =
%A_iL_i^{\beta_i}N_i^{1-\beta_i}
% e^{-\tau_i\tilde z_i}.
%\]
%Aggregate output is a Cobb--Douglas function of land and labor, adjusted with trade costs. The factor is smaller for sectors with large trade costs ($\tau_i$) and sectors farther away from the city (larger $\tilde z_i$). 

Sectoral output per worker is
\[
\frac{\tilde Q_i}{N_i} = \frac1{1-\beta_i}
\frac{W}{P_i}.
\]
We can substitute out product prices in this formula and express them as a function of input costs. To do this, define average sectoral rents as
\begin{equation}
R_i =\int_{z\in Z_i}\frac{L_i(z)}{L_i}R(z)dz = \beta_i(1-\beta_i)^{1/\beta_i-1} \left(P_iA_i\right)^{1/\beta_i} W^{1-1/\beta_i} e^{-\frac{\tau_i}{\beta_i} \tilde{z}_i}
\end{equation}
Note that $R_i$ also equals to the rent prevailing at location $\tilde z_i$. From this, we get that 
%\[
%\frac{R_i}{W} =\beta_i(1-\beta_i)^{1/\beta_i-1} \left(\frac{P_iA_i}{W}\right)^{1/\beta_i} e^{-\frac{\tau_i}{\beta_i} \tilde z_i},
%\]
%where $R_i$ is the rent prevailing at location $\tilde z_i$ (which is also equal to the average rent paid by the sector, as proven below)
%\[
%R(\tilde z_i) =
%R(z_{i-1})e^{-\frac{\tau_i}{\beta_i}(\tilde z_i-z_{i-1})}
%\]

%\[
%\left(\frac{W}{P_i}\right)^{1/\beta_i}  =\beta_i(1-\beta_i)^{1/\beta_i-1}
%A_i^{1/\beta_i}
%\frac{W}{R_i}
% e^{-\frac{\tau_i}{\beta_i} \tilde z_i}
%\]
\[
\frac{W}{P_i}  =\beta_i^{\beta_i}(1-\beta_i)^{1-\beta_i}
A_i
\left(\frac{W}{R_i}\right)^{\beta_i}
 e^{-\tau_i \tilde z_i}
\]
so that output per worker can be written as
\begin{equation}
\frac{\tilde Q_i}{N_i} = \beta_i^{\beta_i}(1-\beta_i)^{-\beta_i}
A_i
\left(\frac{W}{R_i}\right)^{\beta_i}e^{-\tau_i \tilde z_i}.
\end{equation}

Log output per worker is
\[
\tilde q_i - n_i =
\beta_i\ln\beta_i+(1-\beta_i)\ln(1-\beta_i)
+a_i +\beta_i (w
-r_i)
- \tau_{i}\tilde z_{i}
\]
Conditional on true productivity $a_i$, measured productivity (ignoring land) is lower in cities where rents are higher. The magnitude of this bias depends on the (direct and indirect) land share of the sector, $\beta_i$. Measured productivity is also lower whenever the sector locates far from the center ($\tilde z_i$ is high). %Interestingly, this bias does not depend on the land share. The intuition is that XX
The former bias will be bigger for rich countries, as rents tend to be more sensitive to per capita income than wages are. The latter bias will be bigger for rich and urbanized countries, where each urban sector takes up more space. Simply put, services will look unproductive in New York City relative to Budapest, because NYC is larger.

The bias in measured total factor productivity (if location is ignored) depends on sectoral trade costs ($-\tau_i\tilde z_i$), as 
\begin{equation}
\frac{\tilde Q_i}{L_i^{\beta_i}N_i^{1-\beta_i}} = A_i e^{-\tau_i \tilde z_i}.
\end{equation}
The bias is higher in non-tradable sectors, and in large countries and cities. 


\section{Parameter calibration}
We calibrate the model to city-level data from the OECD to show how land usage and sector location varies with development. We fix a number of parameters across cities, but the supply of land and labor, as well as sectoral productivity are allowed to vary.

We assume that cities are in autarky so that we can analyze them separately.\footnote{Desmet and Rossi-Hansberg and XX study a system of cities with endogenous mobility among them.} First we calibrate land shares and shipping costs using US data. Then we turn to the calibration of parameters influencing cross-country, cross-city variations.

\subsection{Land shares}
We calibrate sectoral land shares ($\beta_i$) using US data. Our aim is to capture the share of immobile factors in production. These come from two sources: $(i)$ the direct use of land in production and $(ii)$ the land-rent paid by workers. We calibrate the direct use of land in sectoral production using US factor income share estimates of \citeN{Valentinyi08}. The first two columns of table \ref{tab:Sector_Shares} show their estimates for land and labor shares across sectors.

The indirect use of land is the land used by workers. We calibrate land-rent share in labor as a product of the US aggregate rent-share in consumption expenditure reported by the BLS ($30\%$) and the average land-share of US house prices between 1984-1998 estimated by \citeN{Davis08} (36\%). We find it to be 10.8\%. We multiply this by the labor shares to get the indirect land shares listed in column 3 of table $\ref{tab:Sector_Shares}$. Our calibrated overall land shares ($\beta_i$) are the sum of the direct and indirect land shares, and they are shown in column 4.

The calibrated values show that land is a non-negligible factor in production in each sectors. As expected, its role is the largest in agriculture (23\%), but the land share in manufacturing and services are both double-digit (10\%-13\%), mainly because of the indirect land use of their workers.


\begin{table}[h!]
\label{tab:Sector_Shares}
\caption{Calibrated factor shares}
\begin{center}
\begin{tabular}{l|ccc|c}
\toprule
Factor shares & Direct land & Labor & Indirect land & Overall land share $\beta_i$ \\
\midrule
Agriculture & 0.18 & 0.46  & 0.05 & 0.23 \\
Manufacturing& 0.03 & 0.67 & 0.07 & 0.10  \\
Services    &  0.06 & 0.66 & 0.07 & 0.13 \\
\bottomrule
\end{tabular}
\end{center}

\noindent \footnotesize{Land and Labor shares are estimates of \citeN{Valentinyi08}. Land share in labor is the product of rent-share in US consumption expenditures, the average land-share of US house prices between 1984-1998 estimated by \citeN{Davis08} and the labor shares. Our land share estimates are the sum of direct land share and the indirect land share in labor.}
\end{table}

\subsection{Shipping costs}

We use the 2010 ZIP Business Patters of the U.S. Census to determine the location of sectors in the United States. We use this to calibrate transportation costs and distances of the sectors from the center.

The ZIP Business Patterns contains the number of establishments in employment size categories in each ZIP code for each 6-digit NAICS code. We merge NAICS codes into agriculture, manufacturing and services as follows. Agriculture is sector 11 of NAICS. We merge mining (21), utilities (22), and construction (23) together with manufacturing industries (31-33). As services, we categorize the rest, including public administration. We estimate employment by using the midpoints of the size categories.

To map the model into the data, we need to specify how far each ZIP code is from the city center. We take Urbanized Areas (UAs) as independent monocentric cities, and we assign the central point to the business or administrative center of the first-mentioned city in the UA, as given by Yahoo Maps. For example, the center of ``New York–Newark, NY-NJ-CT Urbanized Area'' is the corner of Broadway and Chamber St in downtown Manhattan, whereas the center of ``Boston, MA–NH-RI Urbanized Area'' is 1 Boston Pl. We calculate the distance of each ZIP code to business center of the nearest UA.

According to equation \ref{eq:EmpDens}, the employment density of sector $i$ in location $z$ is proportional to the rent-wage ratio. When industry $i$ demands positive land in the neighborhood of $z$, then the rent is proportional to $e^{-\tau_i/\beta_i z}$. We can use this observation to estimate $\tau_i$:
\[
\frac{d\ln n_i(z)/l_i(z)}{dz} =\frac{d\ln r_i(z)}{dz} = -\frac{\tau_i}{\beta_i}.
\]

To get a sense where each sectors is active, we calculate location quotients by distance to the city center. For each sector, they measure the employment share at a certain distance relative to the average employment share of the same sector. A value higher than 1 implies that the sector is overrepresented in the particular distance from the center.

\dofigure{sector_location_quotients}{Sector location quotients}{Source: ZIP Business Patterns. Plots the sectoral employment shares at a particular distance (in a 3kms wide ring) from the city center, relative to the average sectoral employment share. The figure shows that sectors sort as in the model: services are overrepresented closer the the city center, while manufacturing and agriculture are located mostly farther away, agriculture showing a particularly steep gradient.}

Figure \ref{fig:sector_location_quotients} shows that sectors sort as in the model: services are overrepresented closer to the city center, while manufacturing and agriculture are both underrepresented there. Agriculture shows a particularly steep gradient, with high relative employment farther away from the city center. In the calculation of sectoral employment gradients below, we restrict attention to the areas where the sectors are overrepresented. We assume that services are active up to 30 kms from the center, manufacturing is active between 10 and 60 kms and agriculture farther than 60 kms.

Let $n_{izc}$ be the employment of industry $i$ in ZIP code $z$, belonging to city (MSA) $c$.  Assuming that establishments in a given sector consume the same amount of land,\footnote{We believe this approximation is likely to bias our estimates of the rent gradient downward. Rural establishments probably occupy more space that urban establishments even in the same narrow industry, so establishment sizes do not go down as fast with distance as employment density does. } we denote by $l_{izc}$ the number of establishments of sector $i$ in ZIP code $z$. We can then regress establishment size (workers per establishment) in each sector in each ZIP code on fixed effects, and the distance of the ZIP code to the city center,
\begin{equation}\label{eq:estimable:gradient}
\frac{n_{izc}}{l_{izc}} = e^{\mu_c+\nu_i-\gamma_i d(z,c)}.
\end{equation}
The city fixed effect captures variation in rents and wages in the MSA, the sector fixed effect captures variation in land and labor intensity and establishment size across sectors. %XX WE MAY NEED CITY*SECTOR FEs
The key parameter of interest is $\gamma_i$, which captures how fast employment declines with distance to the center by sector.

\subsubsection{Imputing employment density at the ZIP-code level}
From the ZBP, we have the approximate employment of the sector (reconstructed from establishment-size bins), and the total area of the ZIP code, but area is not broken down by sector. If a ZIP code is exclusively used by one of the three sector, this is not a problem. Otherwise, we impute the area used by sector $i$ as follows.

The mode predicts the area per worker in sector $i$ in ZIP-code $z$ to be
\[
\frac{L_i(z)}{N_i(z)} = \frac{1-\beta_i}{\beta_i}\frac{W}{R(z)}.
\]
Because all sectors face the same wages and rents in the same ZIP code, we can distribute land in proportion to
\[
\frac{1-\beta_i}{\beta_i}N_i(z).
\]
In urban ZIP codes, there is also a substantial amount of residential land. We know that households spend $0.3\times 0.36$ fraction of their income on residential land rent. Assuming that residents' only income are wages,
\[
\frac{R(z)H(z)}{WP(z)} = 0.3\times 0.36,
\]
where $P(z)$ is the number of people living in ZIP code $z$. Hence total residential area is
\[
H(z) = 0.3\cdot0.36 P(z) \frac{W}{R(z)}.
\]
We then allocate residential land in proportion to $0.3\cdot0.36 P(z)$.

We estimate \eqref{eq:estimable:gradient} by a Poisson regression which ensures that the equation holds in expectation, and permits estimation even when $n_{izc}=0$, which is often the case. The estimates of $\gamma_i$ and the implied sectoral transport cost intensities $\tau_i$ in the three sectors are below.

% Table generated by Excel2LaTeX from sheet 'tau'
\begin{table}[h!]
  \begin{center}
  \caption{Estimated rent and price gradients}
    \begin{tabular}{rccc}
    \toprule
    \textbf{} & \textbf{} & \multicolumn{2}{c}{\textbf{Gradient (per km)}}\\
    \midrule
    \textbf{} & \textbf{Land share $\beta_i$ } & \textbf{Rents $\gamma_i$} & \textbf{Prices $\tau_i$} \\
    Services & 13\%  & 13.15\% & 1.71\% \\
    Manufacturing & 10\%  & 5.15\% & 0.52\% \\
    Agriculture & 23\%  & 3.54\% & 0.81\% \\
    \bottomrule
    \end{tabular}%

  \end{center}
  \label{tab:EmpGrad}%

  \noindent \footnotesize{Sectoral rent gradients $\gamma_i$ are estimated using US employment-density observation across ZIP codes in MSAs. Price gradients reflect transport costs ($\tau_i$) and are estimated by multiplying rent gradients with sectoral land shares ($\beta_i$). }
\end{table}%

The three columns report the calibrated land shares, and the estimates for rent and price gradients, respectively, for each sector. The estimated coefficients can be interpreted as follows. We find that the rents paid by services become 13.15\% cheaper with every kilometer from the city center over the 0-30km range, where services are active. Though a 13\% reduction is substantial, it does not seem unrealistic with a whole 1 kilometer distance. In line with our intuition, rent gradients of manufacturing and agriculture (measured further away from the city center) both imply slower rent declines. We infer price gradients reflecting the transportation cost intensities ($\tau_i$) from the measured rent gradients $\gamma_i=\tau_i/\beta_i$ and the sectoral land shares. The last column of table \ref{tab:EmpGrad} shows that we find transportation costs of services significantly higher (1.71\%) than those of manufacturing and agriculture (0.52\%, 0.81\%). We hold these estimated technology parameters constant across countries. %The employment density gradient of agriculture is inverted, reflecting the fact that most agricultural activity is carried out away from the cities.

\subsection{Utility parameters}
We estimate the utility function \eqref{} in cross-country data from 2007. Expenditure shares ($\{x_i\}$) are from XX. We calculate sectoral price levels ($\{P_i\}$) from the International Comparison Program's XX. We measure per capita income by purchasing power parity adjusted GDP per capita ($y$).

We estimate the following equation for expenditure share of sector $i=1,2,3$ in country $c$ by non-linear least squares,
\begin{equation}
	x_{ic} = 
	\frac 	{\alpha_i \lambda(y_c)^{-\sigma_i}  P_{ic}^{1-\sigma_i}}
			{\sum_j {\alpha_j \lambda(y_c)^{-\sigma_j}  P_{jc}^{1-\sigma_j}}}
	+ \varepsilon_{ic},
\end{equation}
where $\lambda(y_c) = by_c^{-1/\sum_j x_{j0}\sigma_j}$ is the first-order Taylor approximation of marginal utility around average expenditure shares $\{x_{j0}\}$, and $\varepsilon_{ic}$ is an error term independent of income and prices.

Our estimated $\alpha$ and $\sigma$ parameters are shown in Table XX. The model captures the fact that the expenditure share of agriculture is deccreasing in income by estimating a low $\sigma_3$. The income elasticities of manufacturing and services are not significantly different from one another.

$\sigma_1=0.45$, $\sigma_2=0.46$ $\sigma_3=0.16$

\section{Fitting the model to the data}
We use city-level data on land area, employment and GDP per capita from the the OECD Metropolitan Areas database. Our sample includes 277 cities across OECD countries with non-missing area and employment data in 2007.\footnote{We impute employment from population for three cities in Switzerland.}

The following equations completely characterize the spatial equilibrium of city $c$.

(Representative location):
\begin{equation}\label{eq:representative_location}
	\tilde z_{ic}
	=
	- \frac {\beta_i}{\tau_i}
	\ln
	\left[
	\int_{z=z_{i-1,c}}^{z_{ic}}
		\frac {z^2\pi}{L_{ic}}
		\exp(-z \tau_i/\beta_i)
		dz
	\right]
\end{equation}
(Land market clearing):
\begin{equation}\label{eq:land_market_clearing}
	L_{ic}
	=
	\int_{z=z_{i-1,c}}^{z_{ic}}
		z^2\pi
		dz
\end{equation}
with $L_{1c}+L_{2c}+L_{3c}=L_c$.

(Labor intensity):
\begin{equation}\label{eq:labor_intensity}
	\frac 	{N_{ic}}
			{L_{ic}}
	=
	\frac 	{1-\beta_i}
			{\beta_i}
	\exp(\zeta_{ic}-\tilde z_{ic} \tau_i/\beta_i),
\end{equation}
where we introduced
\[
\zeta_{ic}=\ln\beta_i + \frac{1-\beta_i}{\beta_i} \ln (1-\beta_i)
+ \frac 1{\beta_i} (\ln P_{ic} + \ln A_{ic} - \ln W_c)
\]
to simplify notation. The endogenous variable $\zeta_{ic}$ captures relative factor prices in sector $i$.
(Rent arbitrage):
\begin{equation}\label{eq:rent_arbitrage}
	\zeta_{ic} - z_{ic} \tau_i/\beta_i
	=
	\zeta_{i+1,c} - z_{ic} \tau_{i+1}/\beta_{i+1}
\end{equation}
(Labor market clearing):
\begin{equation}\label{eq:labor_market_clearing}
	\sum_i N_{ic} = N_c
\end{equation}
(Goods market clearing):
\begin{equation}\label{eq:goods_market_clearing}
	W_c N_{ic}
	=
	(1-\beta_i)x_{ic}y_c
\end{equation}
For each city $c$, we solve these equations for $(z_1,z_2,z_3)$, $(\tilde z_1,\tilde z_2,\tilde z_3)$, $(L_1,L_2,L_3)$, $(N_1,N_2,N_3)$ and $(\zeta_1,\zeta_2,\zeta_3)$.

\subsection{Recovering city- and sector-level productivities}
\begin{enumerate}
	\item Using the estimated expenditure shares $x_{ic}$ and equations \eqref{eq:goods_market_clearing} and \eqref{eq:labor_market_clearing}, calculate $(N_{1c},N_{2c},N_{3c})$.
	\item Assuming the city is circular, set $z_{3c}=\sqrt{L_c/\pi}$.
	\item Pick a candidate $z_{1c}$ and $z_{2c}$ dividing the service and manufacituring areas.
	\item Using equations \eqref{eq:land_market_clearing} and \eqref{eq:representative_location}, calculate $(L_{1c}, L_{2c}, L_{3c})$ and $(\tilde z_{1c}, \tilde z_{2c}, \tilde z_{3c})$.
	\item Using equation \eqref{eq:labor_intensity}, calculate $(\zeta_{1c}, \zeta_{2c}, \zeta_{3c})$.
	\item If rent arbitrage equation \eqref{eq:rent_arbitrage} is not satisfied, go back to step 3.
	\item Calculate equilibrium wages as
	\[
		W_c = \frac
			{\sum_i (1-\beta_i)Y_{ic}}
			{N_c}
	\]
	and using data on prices, recover
	\[
		A_{ic} = 
			 \frac {\beta_i}{(1-\beta_i)^{1-\beta_i}}
			 \frac {W_c}{P_{ic}}
			 \exp(\zeta_{ic}).
	\]
\end{enumerate}

\subsection{Decomposition of output per worker}
To evaluate the quantitative importance of our mechanism, we first report the decomposition of output per worker \eqref{} into a Hick-neutral productivity term, land per worker, and the representative location of the industry. We express each component relative to Boston, one of the richest cities in the sample.

\begin{table}[h!]
  \begin{center}
  \caption{Distribution of productivity components across cities}
    \begin{tabular}{lccc}
    \toprule
    \textbf{} & \multicolumn{3}{c}{\textbf{Contribution to output per worker of}}\\
    \textbf{} & \textbf{productivity} & \textbf{land per worker} & \textbf{location} \\
    \midrule
        Services 		& 0.194 & 0.957 & 0.950 \\
                 		& 0.712 & 1.266 & 1.092 \\ 
        Manufacturing 	& 0.145 & 0.915 & 0.951 \\
                 		& 0.685 & 1.244 & 1.060 \\
        Agriculture 	& 0.130 & 0.796 & 0.889 \\
                 		& 0.621 & 1.730 & 1.114 \\
    \bottomrule
    \end{tabular}%

  \end{center}
  \label{tab:decomposition}

  \noindent \footnotesize{Notes: Table presents the 10th and 90th percentiles of contributions to output per worker. All contributions are expressed relative to Boston. See text for the precise definition of decomposition.}
\end{table}

Table \ref{tab:decomposition} presents the distribution of productivity components across cities. For each sector and and each productivity component, we report the 10th and 90th percentile of the component relative to Boston. For example, the 10th percentile of Hicks-neutral productivity in the service sector is only 19.4 percent of that in Boston. The productivity gap in similar in the three sectors. 

The contribution of land per worker ranges between 95 and 127 percent for services, 92 and 124 percent for manufacturing, and 80 and 173 percent for agriculture. This is not surprising, as agriculture is the most land intensive of the three sectors and some cities are larger than Boston and have more agricultural land available. 

The contribution of sector location is generally between 90 and 110 percent.

\dofigure{city_level_inputs/productivity}{The contribution of productivity to output per worker}{Notes: Figure presents the nonparametric relationship between city-level GDP per capita and sectoral productivity using locally weighted scatterplot smoothing. Productivity is expressed relative to that in Boston.}

To see how land usage and city structure varies with development, we plot these productivity contributions against GDP per capita of the cities. Figure \ref{fig:city_level_inputs/productivity} shows the nonparametric relationship between productivity and GDP per capita, separately for each of the four sectors. Not surprisingly, richer cities are estimated to have higher productivity. The three sectoral productivities run in parallel, showing that technological differences across cities are unbiased. This is in contrast with XX and XX.

\dofigure{city_level_inputs/land}{The contribution of land usage to output per worker}{Notes: Figure presents the nonparametric relationship between city-level GDP per capita and the contribution of land usage to output per worker using locally weighted scatterplot smoothing. The contribution is expressed relative to that in Boston.}

Figure \ref{fig:city_level_inputs/land} shows the contribution of land per worker to output per worker, relative to Boston. Across the entire level of city GDP per capita, this contribution is greater than one, because most cities are less densely populated than Boston. There is no clear tendency with development, but richer cities tend to have more land avaible for agriculture than for other sectors. This is because agricultural employment in these cities are smaller and they can use larger amounts of land on the city fringe.

\dofigure{city_level_inputs/location}{The contribution of sector location to output per worker}{Notes: Figure presents the nonparametric relationship between city-level GDP per capita and the contribution of sector location to output per worker using locally weighted scatterplot smoothing. The contribution is expressed relative to that in Boston.}

Figure \ref{fig:city_level_inputs/location} plots the contribution of sector location to output per worker. For cities with a GDP per capita below about \$5,000 the contribution of location is greater than in Boston. This is because these cities tend to be smaller in area than Boston, as is shown in Figure \ref{fig:city_level_inputs/city_radius}. For rich and large cities, the spatial sprawl reduces output per worker by about 10 percent in services and manufacturing, and even more in agriculture.

\dofigure{city_level_inputs/city_radius}{City radius and GDP per capita}{Notes: Figure presents the nonparametric relationship between city-level GDP per capita and the radius of the city using locally weighted scatterplot smoothing. City radius is estimated using OECD data on city area and assuming circular cities.}


\section{Conclusion}
We conduct sector-level development accounting in a macro model where land and location play a role. Producers in agriculture, manufacturing and services choose their location endogenously to trade off transport costs and rents. We solve for the spatial equilibrium and show space affects the aggregate production function and measured productivity. Studies not accounting for sector location will deem services in large, expensive cities unproductive. This biases development accounting because rich countries have large service-cities. Our preliminary calibrations show that, correcting for sector location, service productivity varies as much across countries as manufacturing productivity does. This is contrast with previous studies that found smaller variation in service productivity.

XX INTERPRETATION: Back to square 1: we still don’t know which sector drives productivity differences.
2. Variation might come from macro policies and institutions and may not be sector specific.

\clearpage

\bibliography{location}
\bibliographystyle{chicago}


\clearpage

\section{Appendix}

% Table generated by Excel2LaTeX from sheet 'Sheet2'
\begin{table}[htbp]
  \centering
  \caption{Calibrated spatial structures across countries}
    \begin{tabular}{rcccccc}
    \toprule
          & \multicolumn{3}{c}{Share in GDP} & \multicolumn{3}{c}{Distance to center (kms)} \\
    \midrule
    Country & services & manuf & agri  & services & manuf & agri \\
    Austria & 68.39\% & 29.86\% & 1.75\% & 2.5   & 2.6   & 32.4 \\
    Belgium & 75.40\% & 23.72\% & 0.88\% & 15.7  & 16.0  & 20.9 \\
    Canada & 66.77\% & 31.55\% & 1.69\% & 75.3  & 75.7  & 243.0 \\
    Cyprus & 78.82\% & 18.97\% & 2.22\% & 1.3   & 1.3   & 54.2 \\
    Finland & 63.24\% & 33.75\% & 3.01\% & 10.8  & 11.0  & 86.3 \\
    France & 77.17\% & 20.62\% & 2.22\% & 7.0   & 8.3   & 41.1 \\
    Germany & 68.56\% & 30.48\% & 0.96\% & 43.6  & 52.2  & 235.6 \\
    Hungary & 65.78\% & 30.19\% & 4.02\% & 8.2   & 14.5  & 36.9 \\
    Italy & 70.61\% & 27.35\% & 2.05\% & 2.5   & 3.6   & 26.9 \\
    Luxembourg & 83.42\% & 16.18\% & 0.40\% & 3.2   & 6.9   & 28.7 \\
    Netherlands & 73.24\% & 24.68\% & 2.08\% & 6.6   & 6.6   & 13.5 \\
    Poland & 64.04\% & 31.64\% & 4.33\% & 0.2   & 0.5   & 34.0 \\
    Slovenia & 62.90\% & 34.60\% & 2.51\% & 0.7   & 2.6   & 56.6 \\
    United States & 76.95\% & 21.92\% & 1.13\% & 36.1  & 39.3  & 63.5 \\
    \bottomrule
    \end{tabular}%
  \label{tab:share_dist}%
\end{table}%

% Table generated by Excel2LaTeX from sheet 'Sheet1'
\begin{table}[htbp]
  \centering
  \caption{Measured (M'd) and corrected (C'd) sectoral productivities}
    \begin{tabular}{rcccccc}
    \toprule
          & \multicolumn{2}{c}{services} & \multicolumn{2}{c}{manufacturing} & \multicolumn{2}{c}{agriculture}\\
    Country & M'd   & C'd   & M'd   & C'd   & M'd   & C'd \\
    \midrule
    Austria & 0.79  & 0.45  & 0.75  & 0.62  & 0.32  & 0.25 \\
    Belgium & 1.02  & 0.72  & 1.25  & 1.11  & 0.87  & 0.62 \\
    Canada & 1.05  & 2.04  & 1.40  & 1.69  & 0.58  & 2.52 \\
    Cyprus & 1.09  & 0.61  & 0.82  & 0.67  & 0.13  & 0.12 \\
    Finland & 0.75  & 0.49  & 1.08  & 0.94  & 0.21  & 0.25 \\
    France & 0.93  & 0.57  & 0.79  & 0.67  & 0.44  & 0.37 \\
    Germany & 1.02  & 1.16  & 0.84  & 0.90  & 0.50  & 2.03 \\
    Greece & 1.04  & 0.00  & 0.58  & 0.57  & 0.12  & 0.09 \\
    Hungary & 0.55  & 0.34  & 0.31  & 0.27  & 0.20  & 0.16 \\
    Italy & 0.94  & 0.53  & 1.01  & 0.84  & 0.25  & 0.19 \\
    Luxembourg & 1.92  & 1.10  & 0.81  & 0.69  & 0.52  & 0.39 \\
    Netherlands & 1.03  & 0.62  & 1.73  & 1.46  & 0.61  & 0.41 \\
    Poland & 0.75  & 0.41  & 0.36  & 0.30  & 0.07  & 0.06 \\
    Slovenia & 0.62  & 0.34  & 0.41  & 0.34  & 0.07  & 0.06 \\
    United States & 1.00  & 1.00  & 1.00  & 1.00  & 1.00  & 1.00 \\
    \bottomrule
    \end{tabular}%
  \label{tab:prods}%
\end{table}%

\dofigure{measured_prod_1}{Measured productivity in services}{}
\dofigure{true_prod_1}{Location-corrected productivity in services}{}


\dofigure{measured_prod_2}{Measured productivity in manufacturing}{}
\dofigure{true_prod_2}{Location-corrected productivity in manufacturing}{}


\dofigure{measured_prod_3}{Measured productivity in agriculture}{}
\dofigure{true_prod_3}{Location-corrected productivity in agriculture}{}


\end{document}
