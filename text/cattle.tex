\documentclass[12pt]{article}
\usepackage{amsmath,amsthm}
\usepackage[utf8]{inputenc}
\usepackage[T1]{fontenc}
%\usepackage{minionpro}
\usepackage{array,graphicx}

\setlength{\topmargin}{-0.3in} \setlength{\textheight}{8.75in}
\setlength{\oddsidemargin}{0.0in} \setlength{\evensidemargin}{0.0in}
\setlength{\textwidth}{6.5in}
\def\labelenumi{\arabic{enumi}.}
\def\theenumi{\arabic{enumi}}
\def\labelenumii{(\alph{enumii})}
\def\theenumii{\alph{enumii}}
\def\p@enumii{\theenumi.}
\def\labelenumiii{\arabic{enumiii}.}
\def\theenumiii{\arabic{enumiii}}
\def\p@enumiii{(\theenumi)(\theenumii)}
\def\labelenumiv{\arabic{enumiv}.}
\def\theenumiv{\arabic{enumiv}}
\def\p@enumiv{\p@enumiii.\theenumiii}
\pagestyle{plain}
\pagestyle{plain} \setcounter{secnumdepth}{3}
\newcommand{\D}{\mathop{\mathrm{d\mathstrut}}\nolimits\!}
\newcommand{\dt}{\D t}
\newcommand{\dz}{\D z}
\newcommand{\E}{\mathop{\mathrm{E\mathstrut}}\nolimits}
\newcommand{\Var}{\mathop{\mathrm{Var\mathstrut}}\nolimits}
\newcommand{\sd}{\mathop{\mathrm{sd\mathstrut}}\nolimits}
\newcommand{\diag}{\mathop{\mathrm{diag\mathstrut}}\nolimits}
\newcommand{\Cov}{\mathop{\mathrm{Cov\mathstrut}}\nolimits}
\newcommand{\Corr}{\mathop{\mathrm{Corr\mathstrut}}\nolimits}
\newtheorem{definition}{Definition}
\newtheorem{proposition}{Proposition}
\newtheorem{conjecture}{Conjecture}
\newtheorem{moment}{Empirical regularity}
\newtheorem{insight}{Qualitative prediction}

\newcommand{\dofigure}[2]{\begin{figure}
\begin{centering}
\includegraphics[width=0.75\linewidth]{figures/#1}
  \caption{#2\label{fig:#1}}
\end{centering}
\end{figure}}

\newcommand{\dotable}[2]{\begin{table}[h!]
\begin{centering}
\caption{#2\label{tab:#1}}
\includegraphics[width=0.75\linewidth]{figures/#1}
\end{centering}
\end{table}}

\begin{document}

\title{Cattle, Steaks and Restaurants: Development Accounting when Space Matters}
\author{Péter Karádi\thanks{European Central Bank. E-mail: peter.karadi@ecb.int}~ and Miklós Koren\thanks{Central European University, IE--HAS and CEPR. E-mail: korenm@ceu.hu.}}
\maketitle

\begin{abstract}
\end{abstract}

Productivity is much lower in poor countries than in rich ones (REFS). To understand the fundamental causes of productivity differences, it is important to identify the sectors in which these differences are greatest (REF). Several recent papers have studied the sectoral composition of productivity differences by using data on sector-level prices (REF). The main result is XXX. Intuitively, prices are XXX

We revisit the measurement of sectoral prices and sectoral productivity in a macro model where land and location play a role. A high price in one sector may simply reflect high rents accruing to a non-reproducible input, land. For example, the fact that restaurants are expensive in New York City has more to with high rents, and may not imply that NYC restaurants are inefficient. Rents are, in turn, determined in general equilibrium, and may respond to demand for land in other activities, such as finance, culture and housing in the NYC example.

We build a multi-sector general equilibrium model. Each sector uses labor (or a composite of other spatially mobile inputs) and land. The location of sectors is determined in the canonical von Thünen city model. XXX DESCRIBE MODEL

Our model yields a simple spatial equilibrium in which agriculture (``cattle'') locates farthest away from the center, manufacturing (``steaks'') occupies a ring outside the center, and services (``restaurants'') are in a central circle.

There are two reasons why development accounting in our model is different from models without land. First, as some sectors are more land intensive, their prices may be more sensitive to rents. Because of this, conditional on productivity, agricultural prices will be relatively higher in rich countries. Second, because sectors endogenously choose locations, their price is also affected by the rent gradient: the speed with which rents decline in distance from the city center. Urban sectors will be relatively more expensive in countries where the rent gradient is higher.

To quantify the importance of these two mechanisms, we calibrate our model to match the spatial distribution of economic activities in the U.S., and the share of sectors in each country. We then use the calibrated model to infer sector-level productivities for each country using data on relative prices (XX AND URBANIZATION)

Our main findings are as follows. XX


\section{A model of industry location}
\subsection{Technology}
Output in sector $i$ at location $z$ depends on employment and land used at that location,
\[
Q_i(z) = A_i L_i(z)^{\beta_i}N_i(z)^{1-\beta_i}.
\]
We take labor to be freely mobile within the country, land is in fixed supply in each location. The names ``land'' and ``labor'' are for the sake of convenience, these two factors correspond to spatially fixed and mobile factors, respectively, and we will calibrate them accordingly.

Sectors differ in their land shares $\beta_i$ and Hicks neutral productivity shifter $A_i$.

All products are sold and consumed at a single location, the central business district. This is location $z=0$, so that $z$ indexes distance to the center.

\subsection{Shipping}
To ship a product to the center, one has to incur shipping costs. If a unit of product $i$ leaves location $z$, only
\[
e^{-\tau_i z}
\]
units arrive at the center. This is akin to the iceberg assumption of fixed costs. Sectors also differ in the intensity of shipping costs $\tau_i$.

\subsection{Consumers}
Consumers have Cobb--Douglas preferences across the three goods,
\[
U = C_1^{\alpha_1}C_2^{\alpha_2}C_3^{\alpha_3}
\]


\subsection{Spatial structure}
At each location $z$ where industry $i$ is active, the value marginal product of land and labor have equal rents and wages, respectively.
\begin{align}
R(z) &=\beta_i P_ie^{-\tau_i z}A_i \left(\frac{N_i(z)}{L_i(z)}\right)^{1-\beta_i}\\
W &=(1-\beta_i) P_ie^{-\tau_i z}A_i \left(\frac{N_i(z)}{L_i(z)}\right)^{-\beta_i}
\end{align}
Because labor is freely mobile, wages do not depend on location.

The labor-land ratio (employment density) can be expressed as
\[
\frac{N_i(z)}{L_i(z)} = \frac{1-\beta_i}{\beta_i}\frac{R(z)}{W}.
\]
Substituting this into the FOC for land,
\[
R(z) =\beta_i P_ie^{-\tau_i z}A_i \left(\frac{1-\beta_i}{\beta_i}\frac{R(z)}{W}\right)^{1-\beta_i}
\]
\[
R(z)^{\beta_i} =\beta_i^{\beta_i}(1-\beta_i)^{1-\beta_i} P_ie^{-\tau_i z}A_i W^{\beta_i-1}
\]
\[
R(z) =\beta_i(1-\beta_i)^{1/\beta_i-1} (P_iA_i)^{1/\beta_i} W^{1-1/\beta_i} e^{-\frac{\tau_i}{\beta_i} z}
\]
This pins down the gradient of the bid-rent curve. Substituting into the labor-land ratio,
\[
\frac{N_i(z)}{L_i(z)} = (1-\beta_i)^{1/\beta_i} \left(\frac{P_iA_i}{W}\right)^{1/\beta_i} e^{-\frac{\tau_i}{\beta_i} z}
\]
gives us the employment gradient.

\subsection{Aggregation across space}
Value added per unit of land at consumer prices,
\[
\frac{P_i Q_i(z)}{L_i(z)} = (1-\beta_i)^{1/\beta_i-1}
(P_iA_i)^{1/\beta_i}W^{1-1/\beta_i}
 e^{-\frac{1-\beta_i}{\beta_i}\tau_i z}.
\]
The same at producer prices
\[
\frac{e^{-\tau_i z} P_i Q_i(z)}{L_i(z)} = (1-\beta_i)^{1/\beta_i-1}
(P_iA_i)^{1/\beta_i}W^{1-1/\beta_i}
 e^{-\frac{\tau_i}{\beta_i} z}.
\]
Let 
\[
\Lambda_i(z) = \int_0^z L(s)e^{-\frac{\tau_i}{\beta_i} s}ds
\]
denote the cumulative area between $0$ and $z$ weighted by the rent gradient of sector $i$. The the total supply of services and manufacturing at producer prices,
\begin{align*}
Y_1 &= (1-\beta_1)^{1/\beta_1-1}
(P_1A_1)^{1/\beta_1}W^{1-1/\beta_1} \Lambda_1(z_1)\\
Y_2 &= (1-\beta_2)^{1/\beta_2-1}
(P_2A_2)^{1/\beta_2}W^{1-1/\beta_2} [\Lambda_2(z_2)-\Lambda_2(z_1)]
\end{align*}
Overall employment in the sectors,
\begin{align*}
N_1 &= (1-\beta_1)^{1/\beta_1}
\left(\frac{P_1A_1}{W}\right)^{1/\beta_1} \Lambda_1(z_1)\\
N_2 &= (1-\beta_2)^{1/\beta_2}
\left(\frac{P_2A_2}{W}\right)^{1/\beta_2}  [\Lambda_2(z_2)-\Lambda_2(z_1)]
\end{align*}

\[
N_i = (1-\beta_i)^{1/\beta_i}
\left(\frac{P_iA_i}{W}\right)^{1/\beta_i}
\int_{z\in Z_i}L_i(z) e^{-\frac{1}{\beta_i}\tau_i z}dz
\]
so that we can express $W/P_i$ as
\[
\left(\frac{W}{P_i}\right)^{1/\beta_i} = (1-\beta_i)^{1/\beta_i}
N_i^{-1}A_i^{1/\beta_i}
\int_{z\in Z_i}L_i(z) e^{-\frac{1}{\beta_i}\tau_i z}dz
\]
\[
\frac{W}{P_i} = (1-\beta_i)
N_i^{-\beta_i}A_i
\left[\int_{z\in Z_i}L_i(z) e^{-\frac{1}{\beta_i}\tau_i z}dz\right]^{\beta_i}
\]
\[
\left(\frac{W}{P_i}\right)^{1-1/\beta_i} = (1-\beta_i)^{1-1/\beta_i}
N_i^{1-\beta_i}A_i^{1-1/\beta_i}
\left[\int_{z\in Z_i}L_i(z) e^{-\frac{1}{\beta_i}\tau_i z}dz\right]^{\beta_i-1}
\]

\subsection{Productivity measurement}
Suppose physical output is measured as revenue at producer prices deflated by a common price index. We take the price index to be the consumer price (XX CHECK IF IT MATTERS).
\[
\tilde Q_i(z) = \frac{e^{-\tau_i z}P_iQ_i(Z)}{P_i}
\]
We can write an aggregate output as
\[
\tilde Q_i = 
(1-\beta_i)^{1/\beta_i-1}
A_i^{1/\beta_i}\left(\frac{W}{P_i}\right)^{1-1/\beta_i}
\int_{z\in Z_i}L_i(z) e^{-\frac{1}{\beta_i}\tau_i z}dz
\]
which takes us to the aggregate production function
\[
\tilde Q_i = 
A_iN_i^{1-\beta_i}\left[\int_{z\in Z_i}L_i(z) e^{-\frac{1}{\beta_i}\tau_i z}dz\right]^{\beta_i}\]
\[
\tilde Q_i = 
A_iL_i^{\beta_i}N_i^{1-\beta_i}
\left[\int_{z\in Z_i}\frac{L_i(z)}{L_i} e^{-\frac{1}{\beta_i}\tau_i z}dz\right]^{\beta_i}
\]
where $L_i$ is the unweighted amount of land devoted to the production of good $i$. Aggregate output is a Cobb--Douglas function of land and labor, adjusted with trade costs (the last two terms). The adjustment factor is always less than one, as it is a weighted average of trade cost factors. The factor is smaller for sectors with large trade costs ($\tau_i$) and sectors farther away from the city. Simply put, services will look unproductive in New York City relative to Budapest, because NYC is larger.

Output per worker
\[
\frac{\tilde Q_i}{N_i} = \frac1{1-\beta_i}
\frac{W}{P_i} 
\]
We want to get rid of prices in this formula
\[
\frac{R(z)}{W} =\beta_i(1-\beta_i)^{1/\beta_i-1} \left(\frac{P_iA_i}{W}\right)^{1/\beta_i} e^{-\frac{\tau_i}{\beta_i} z}
\]
\[
\left(\frac{W}{P_i}\right)^{1/\beta_i}  =\beta_i(1-\beta_i)^{1/\beta_i-1} 
A_i^{1/\beta_i}
\frac{W}{R(z)}
 e^{-\frac{\tau_i}{\beta_i} z}
\]
\[
\frac{W}{P_i}  =\beta_i^{\beta_i}(1-\beta_i)^{1-\beta_i} 
A_i
\left(\frac{W}{R(z)}\right)^{\beta_i}
 e^{-\tau_i  z}
\]
\[
\frac{\tilde Q_i}{N_i} = \beta_i^{\beta_i}(1-\beta_i)^{-\beta_i} 
A_i
\left[\frac{W}{R_i(0)}\right]^{\beta_i}
 \]
where $R_i(0)$ is the intercept of the rent gradient of sector $i$, that is, the amount of rent it would be willing to pay at the center. 

For $i>0$, we can write 
\[
R_i(0) \equiv R_i(z_{i-1})e^{\frac{\tau_i}{\beta_i}z_{i-1}} = R_{i-1}(z_{i-1})e^{\frac{\tau_i}{\beta_i}z_{i-1}}
= R_{i-1}(z_{i-2})
e^{-\frac{\tau_{i-1}}{\beta_{i-1}}(z_{i-1}-z_{i-2})+\frac{\tau_i}{\beta_i}z_{i-1}}
\]
so that
\[
r_i(0) = r(0)-\sum_{k=1}^i 
\frac{\tau_{k}}{\beta_{k}}(z_{k}-z_{k-1})
+ \frac{\tau_{i}}{\beta_{i}}z_{i}
\]
Log output per worker is
\[
\tilde q_i - n_i = 
\beta_i\ln\beta_i+(1-\beta_i)\ln(1-\beta_i)
+a_i +\beta_i [w
-r(0)]
+\sum_{k=1}^i 
\frac{\beta_i}{\beta_{k}}\tau_{k}(z_{k}-z_{k-1})
- \tau_{i}z_{i}
\]
Conditional on true productivity $a_i$, measured productivity is lower in cities where rents are higher. The magnitude of this bias depends on the (direct and indirect) land share of the sector, $\beta_i$. Measured productivity is also lower whenever the sector locates far from the center ($z_i$ is high). Interestingly, this bias does not depend on the land share. The intuition is that XX

XX WE HAVE YET TO INTERPRET THE MIDDLE TERM

The former bias will be bigger for rich countries, as rents tend to be more sensitive to per capita income than wages are. The latter bias will be bigger for rich and urbanized countries, where each urban sector takes up more space.

\subsection{Development accounting}

Inferring productivity from output per worker,
\[
a_i =  \tilde a_i +\gamma_i
+\beta_i [r(0)-w]
-\sum_{k=1}^i 
\frac{\beta_i}{\beta_{k}}\tau_{k}(z_{k}-z_{k-1})
+ \tau_{i}z_{i},
\]
where $\tilde a_i = q_i-n_i$ is log measured productivity and $\gamma_i = -\beta_i\ln\beta_i-(1-\beta_i)\ln(1-\beta_i)$ is a sector-specific constant. This is our main equation for sector-level development accounting.

We are interested in how sectoral productivities are correalted with per capita income.
\[
\frac{d a_i}{dy_i} = \frac{d \tilde a_i}{dy_i} +\beta_i \left[
\frac{d r(0)}{dy_i}-\frac{d w}{dy_i}\right]
-\sum_{k=1}^i 
\frac{\beta_i}{\beta_{k}}\tau_{k}\left[\frac{d z_k}{dy_i}-\frac{d z_{k-1}}{dy_i}\right]
+ \tau_{i}\frac{d z_i}{dy_i}.
\]


\subsection{Spatial arbitrage}
Competitive land markets ensure that each location goes to the highest bidder. The slope of the bid-rent gradient in sector $i$ is $\tau_i/\beta_i$. Suppose that this is the highest for services, lower for manufacturing, and lowest for agriculture.

Let $z_i$ denote the outer edge of the land use of sector $i$. At such locations, both sector $i$ and sector $i+1$ have the same reservation rent: 
\begin{align*}
R(z_i) &=\beta_i(1-\beta_i)^{1/\beta_i-1} (P_iA_i)^{1/\beta_i} W^{1-1/\beta_i} e^{-\frac{\tau_i}{\beta_i} z_i}\\
R(z_i) &=\beta_{i+1}(1-\beta_{i+1})^{1/\beta_{i+1}-1} (P_{i+1}A_{i+1})^{1/\beta_{i+1}} W^{1-1/\beta_{i+1}} e^{-\frac{\tau_{i+1}}{\beta_{i+1}} z_i}
\end{align*}

Substitute in the overall amount of value added,
\[
R_i(z) = \frac{\beta_i Y_i}{\Lambda_i(z_i)-\Lambda_i(z_{i-1})} e^{-\frac{\tau_{i}}{\beta_{i}} z}.
\]


Let $\underline R$ denote the rent prevalining beyond the urban fringe (we may want to make agricuture costless to trade). The we can find the urban fringe as the location $z_2$ such that $R_2(z_2)=\underline R$.

Agricultural rent is
\[
\underline R = \frac{\beta_3 Y_3}{\int_{z_2}^{\bar z}L(s)ds} 
\]

\[
 \frac{\beta_3 Y_3}{\int_{z_2}^{\bar z}L(s)ds} =
 \frac{\beta_2 Y_2}{\Lambda_2(z_2)-\Lambda_2(z_1)} e^{-\frac{\tau_{2}}{\beta_{2}} z_2}
\]
\[
\frac{\tau_2}{\beta_2}z_2 =
\ln\beta_2-\ln\beta_3 +y_2-y_3+\ln\left[\int_{z_2}^{\bar z}L(s)ds\right]
-\ln[\Lambda_2(z_2)-\Lambda_2(z_1)]
\]

The services-manufacturing barrier,
\[
 \frac{\beta_1 Y_1}{\Lambda_1(z_1)} e^{-\frac{\tau_{1}}{\beta_{1}} z_1} =
 \frac{\beta_2 Y_2}{\Lambda_2(z_2)-\Lambda_2(z_1)} e^{-\frac{\tau_{2}}{\beta_{2}} z_1}
\]

\[
e^{\left(\frac{\tau_{1}}{\beta_{1}}-\frac{\tau_{2}}{\beta_{2}}\right) z_1}
= 
 \frac{\beta_1 Y_1}{\beta_2 Y_2}  
 \frac{\Lambda_2(z_2)-\Lambda_2(z_1)}{\Lambda_1(z_1)} 
\]

The following two equations implicitly define the two barriers $z_1$ and $z_2$.
\begin{align}
 \frac{\beta_1 Y_1}{\beta_2 Y_2}  
 &=
 e^{\left(\frac{\tau_{1}}{\beta_{1}}-\frac{\tau_{2}}{\beta_{2}}\right) z_1}
 \frac{\Lambda_1(z_1)}{\Lambda_2(z_2)-\Lambda_2(z_1)} 
\\
 \frac{\beta_2 Y_2}  {\beta_3 Y_3}
 &=
 e^{\frac{\tau_{2}}{\beta_{2}} z_2}
 \frac{\Lambda_2(z_2)-\Lambda_2(z_1)} {\int_{z_2}^{\bar z}L(s)ds}
\end{align}


\section{Data and calibration}

\subsection{Variation across countries}
Countries differ in Hicks-neutral productivities $A_i$, their urban fringe $z_3$ and their demand parameters $\alpha_i$. Each country has the same land share within sectors $\beta_i$ and the same shipping cost $\tau_i$.

We allow for sector-specific differences in productivity, which are the key object of interest in multi-sector development accounting. The urban fringe may also vary with country area and its degree of urbanization and we calibrate it below. Demand parameters are allowed to vary to allow different countries to have different expenditure shares in agriculture, manufacturing and services. XX TALK MORE ABOUT STRUCTURAL CHG

\subsection{Calibration}

We calibrate sectoral land shares ($\beta_i$) using US data. Our aim is to capture the share of immobile factors in production. These come from two sources: $(i)$ the direct use of land in production and $(ii)$ the land-rent paid by workers. We calibrate the direct use of land in sectoral production using US factor income share estimates of Herrendorf and Valentinyi (2008). The first two columns of table XX show their estimates for land and labor shares across sectors.

Our land-rent share in labor estimate is a product of the US aggregate rent-share in consumption expenditure reported by the BLS ($30\%$) and the average land-share of US house prices between 1984-1998 estimated by Davis and Palumbo (2008) (36\%). We multiply this product by the labor shares, and add this product to the direct land use to obtain our land share estimate.

\begin{table}[h!] \center
\begin{tabular}{l|ccc|c}
\hline 
Factor shares & Direct land & Labor & Indirect land & Overall land share $\beta_i$ \\ \hline
Agriculture & 0.18 & 0.46  & 0.05 & 0.23 \\
Manufacturing& 0.03 & 0.67 & 0.07 & 0.10  \\
Services    &  0.06 & 0.66 & 0.07 & 0.13 \\ \hline 
\end{tabular}

\noindent \footnotesize{Land and Labor shares are estimates of Herrendorf and Valentinyi (2008). Land share in labor is the product of rent-share in US consumption expenditures, the average land-share of US house prices between 1984-1998 estimated by Davis and Palumbo (2008) and the labor shares. Our land share estimates are the sum of direct land share and the indirect land share through labor.}
\end{table}





\subsection{Sector gradients}
We use the 2010 ZIP Business Patters of the U.S. Census to determine the location of sectors in the United States. We use this to calibrate transportation costs and XX.

The ZIP Business Patterns contains the number of establishments in employment size categories in each ZIP code for each 6-digit NAICS code. We merge NAICS codes into agriculture, manufacturing and services as follows. XX TO DO.

To map the model into the data, we need to specifiy how far each ZIP code is from the city center. We take Urbanized Areas (UAs) as independent monocentric cities, and we assign the central point to the business or administrative center of the first-mentioned city in the UA, as given by Yahoo Maps. For example, the center of ``New York–Newark, NY-NJ-CT Urbanized Area'' is the corner of Broadway and Chamber St in downtown Manhattan, whereas the center of ``Boston, MA–NH-RI Urbanized Area'' is 1 Boston Pl.

We calulcate the distance of each ZIP code to business center of the nearest UA.

According to the model XX, the employment density of sector $i$ in location $z$ proportional to the rent-wage ratio. When industry $i$ demands positive land in the neighborhood of $z$, then the rent gradient is proportional to $e^{-\tau_iz}$. We can use this observation to estimate $tau_i$.
\[
\frac{d\ln n(z)/l(z)}{dz} =\frac{d\ln r(z)}{dz} = -\tau_i.
\]
Let $n_{izc}$ be the employment of industry $i$ in ZIP code $z$, belonging to city (MSA) $c$. (Note that this employment can, and very often will be zero.) We denote the land area of ZIP code $z$ by $l_z$.
\begin{equation}\label{eq:estimable:gradient}
\frac{n_{izc}}{l_z} = e^{\mu_c+\nu_i-\tau_i d(z,c)}
\end{equation}
The city fixed effect captures variation in rents and wages in the MSA, the sector fixed effect captures variation in land and labor intensity across sectors. XX WE MAY NEED CITY*SECTOR FEs The key parameter of interest is $\tau$, which captures how fast employment declines with distance to the center.

We estimate \eqref{eq:estimable:gradient} by a Poisson regression which ensures that the equation holds in expectation, and permits estimation even when $n_{izc}=0$, which is often the case. The estimates of $\tau_i$ in the three sectors are below. XX TO DO

XX TABLE HERE

The three columns report the estimates for agriculture, manufacturing and services, respectively. The estimated coefficients can be interpreted as follows. The employment density gradient of agriculture is inverted, reflecting the fact that most agricultural activity is carried out away from the cities.

\subsection{Sector locations}
We calibrate $\tau$ so as to match the location of sectors in the U.S. We use the Zip Business Patterns data (XX COPY FROM ABOVE) to calculate the employment-weighted median distance of each sector to the center of the MSA.

Figure XX shows the empirical distribution of distance across the three sectors. While there is certainly a wide range of distances for each sector, and the location of sectors overlaps, there is also a clear location shift visible: the typical agricultural employee is farther away than the typical industry employee, who is, in turn, farther away that service employees.

Table XX reports the median distance of each sector to the city center. Services are closest with 15.9km, manufacturing are 20.2km, whereas agriculture is 38.5km away from the center, as weighted by employment.

\[
\frac{\int_0^z Y_1(s)ds}{\int_0^{z_1} Y_1(s)ds} = \frac{\Lambda_1(z)}{\Lambda_1(z_1)}
\]

\subsection{The urban fringe}



\end{document}

