\documentclass[handout,compress,mathserif]{beamer}
\usepackage[latin2]{inputenc}
%\usepackage[absolute]{textpos}
%\documentclass[handout,compress,mathserif]{beamer}
%\setbeameroption{show notes}

% This file is a solution template for:

% - Talk at a conference/colloquium.
% - Talk length is about 20min.
% - Style is ornate.



% Copyright 2004 by Till Tantau <tantau@users.sourceforge.net>.
%
% In principle, this file can be redistributed and/or modified under
% the terms of the GNU Public License, version 2.
%
% However, this file is supposed to be a template to be modified
% for your own needs. For this reason, if you use this file as a
% template and not specifically distribute it as part of a another
% package/program, I grant the extra permission to freely copy and
% modify this file as you see fit and even to delete this copyright
% notice.


\mode<presentation>
{
  %\usetheme{pittsburgh}
  % or ...

  \setbeamercovered{invisible}
  % or whatever (possibly just delete it)
}


\usepackage[USenglish]{babel}
%\usepackage[latin1]{inputenc}
\usepackage[T1]{fontenc}
\usepackage{mathpazo}
\usepackage{ifthen,array}

\pretolerance5000 \hyphenpenalty9999
%\setlength{\TPHorizModule}{0.5cm} \setlength{\TPVertModule}{0.5cm}
%\textblockorigin{20mm}{20mm} % start everything near the top-left corner

\newcounter{ora}
\newcounter{perc}
\newcounter{kezdoora}
\newcounter{kezdoperc}
\newcounter{percek}
\setcounter{percek}{0}
\setcounter{kezdoora}{12} % for 1.35pm as the starting time

\providecommand{\leadingzero}[1]{\ifthenelse{\value{#1}<10}{0\arabic{#1}}{\arabic{#1}}}
\providecommand{\oradisplay}[1]{\ifthenelse{\value{#1}<60}{\arabic{kezdoora}:\leadingzero{#1}}{\setcounter{perc}{\value{#1}}\addtocounter{perc}{-60}\setcounter{ora}{\value{kezdoora}}\addtocounter{ora}{1}\arabic{ora}:\leadingzero{perc}}}

\providecommand{\notes}[1]{{\tiny\textbf{Note:} #1}}
%%%%%%%%%%%%%%%%%%%%%%%%%%%%%%%%%%%%%%%%%%%%%%%%
%% Hasznos matek makrok
%%%%%%%%%%%%%%%%%%%%%%%%%%%%%%%%%%%%%%%%%%%%%%%%

\newcommand{\QED}{{}\hfill$\Box$}
\newcommand{\intl}[4]{\int_{#1}^{#2} \! {#3} \, \mathrm d{#4}}
\newcommand{\period}{\text{.}} % Ez azert kell, mert a matek . mashogy nez ki, mint a szovege.
\newcommand{\comma}{\text{,}}  % Ez azert kell, mert a matek , mashogy nez ki, mint a szovege.
\newcommand{\dist}{\,\mathop{\operatorname{\sim\,}}\limits}
\newcommand{\D}{\,\mathop{\operatorname{d}}\!}
%\newcommand{\E}{\mathop{\operatorname{E}}\nolimits}
\newcommand{\Lag}{\mathop{\operatorname{L}}}
\newcommand{\plim}{\mathop{\operatorname{plim}}\limits_{T\to\infty}\,}
\newcommand{\CES}[3]{\mathop{\operatorname{CES}}\left(\left\{#1\right\},\left\{#2\right\},#3\right)}
\newcommand{\cestwo}[5]{\left[#1^\frac1{#5}\,#2^\frac{#5-1}{#5}+#3^\frac1{#5}\,#4^\frac{#5-1}{#5}\right]^\frac{#5}{#5-1}}
\newcommand{\cesmore}[4]{\left[\sum_{#3}#1_{#3}^\frac1{#4}\,{#2}_{#3}^\frac{#4-1}{#4}\right]^\frac{#4}{#4-1}}
\newcommand{\cesPtwo}[5]{\left[#1\,#2^{1-#5}+#3\,#4^{1-#5}\right]^\frac{1}{1-#5}}
\newcommand{\cesPmore}[4]{\left[\sum_{#3}#1_{#3}\,#2_{#3}^{1-#4}\right]^\frac{1}{1-#4}}
\newcommand{\diff}[2]{\frac{\D #1}{\D #2}}
\newcommand{\pdiff}[2]{\frac{\partial #1}{\partial #2}}
\newcommand{\convex}[2]{\lambda #1 + (1-\lambda)#2}
\newcommand{\ABS}[1]{\left| #1 \right|}
\newcommand{\suchthat}{:\hskip1em}
\newcommand{\dispfrac}[2]{\frac{\displaystyle #1}{\displaystyle #2}} % Emeletes tortekhez hasznos.

\newcommand{\diag}{\mathop{\mathrm{diag\mathstrut}}}
\newcommand{\tr}{\mathop{\mathrm{tr\mathstrut}}}
\newcommand{\E}{\mathop{\mathrm{E\mathstrut}}}
\newcommand{\Var}{\mathop{\mathrm{Var\mathstrut}}\nolimits}
\newcommand{\Cov}{\mathop{\mathrm{Cov\mathstrut}}}
\newcommand{\sgn}{\mathop{\operatorname{sgn\mathstrut}}}

\newcommand{\covmat}{\mathbf\Sigma}
\newcommand{\ones}{\mathbf 1}
\newcommand{\zeros}{\mathbf 0}
\newcommand{\BAR}[1]{\overline{#1}}


\newlength{\tempsep}

\newenvironment{subeqs}{\setlength{\tempsep}{\arraycolsep}
\setlength{\arraycolsep}{0.13889em} % Ez azert kell, hogy ne hagyjon tul sok helyet az = korul.
\begin{subequations}\begin{eqnarray}}
{\end{eqnarray}\end{subequations}
\setlength{\arraycolsep}{\tempsep}}

\newenvironment{tapad}{\setlength{\tempsep}{\arraycolsep}
\setlength{\arraycolsep}{0.13889em}} % Ez azert kell, hogy ne hagyjon tul sok helyet az = korul.
{\setlength{\arraycolsep}{\tempsep}}

\newenvironment{eqnarr}{\setlength{\tempsep}{\arraycolsep}
\setlength{\arraycolsep}{0.13889em} % Ez azert kell, hogy ne hagyjon tul sok helyet az = korul.
\begin{eqnarray}}
{\end{eqnarray} \setlength{\arraycolsep}{\tempsep}}

\newenvironment{eqnarr*}{\setlength{\tempsep}{\arraycolsep}
\setlength{\arraycolsep}{0.13889em} % Ez azert kell, hogy ne hagyjon tul sok helyet az = korul.
\begin{eqnarray*}}
{\end{eqnarray*} \setlength{\arraycolsep}{\tempsep}}


%\usepackage[active]{srcltx} % SRC Specials: DVI [Inverse] Search
% Fuzz --- -------------------------------------------------------
\hfuzz5pt % Don't bother to report over-full boxes < 5pt
\vfuzz5pt % Don't bother to report over-full boxes < 5pt
% THEOREMS -------------------------------------------------------
% MATH -----------------------------------------------------------
\newcommand{\norm}[1]{\left\Vert#1\right\Vert}
\newcommand{\abs}[1]{\left\vert#1\right\vert}
\newcommand{\set}[1]{\left\{#1\right\}}
\newcommand{\Real}{\mathbb R}
\newcommand{\eps}{\varepsilon}
\newcommand{\To}{\longrightarrow}
\newcommand{\BX}{\mathbf{B}(X)}
\newcommand{\A}{\mathcal{A}}

%\renewcommand{\rmdefault}{pmnr}
\renewcommand{\sfdefault}{iwona}



\newcommand{\directory}{figures}
\newcommand*{\newtitle}{\egroup\begin{frame}\frametitle}

\newcommand{\fullpagefigure}[2]{\begin{frame}\frametitle{\hyperlink{#1back}{#2}}\hypertarget{#1}{{\begin{center}\includegraphics[height=0.9\textheight]{\directory/#1}\end{center}}}\end{frame}}
\newcommand{\widefigure}[2]{\begin{frame}\frametitle{\hyperlink{#1back}{#2}}\hypertarget{#1}{{\begin{center}\includegraphics[width=\linewidth]{\directory/#1}\end{center}}}\end{frame}}
\newcommand{\longfigure}[2]{\begin{frame}\frametitle{\hyperlink{#1back}{#2}}\hypertarget{#1}{{\begin{center}\includegraphics[height=0.8\textheight]{\directory/#1}\end{center}}}\end{frame}}
%\newcommand{\fullpagefigure}[2]{\begin{frame}\frametitle{\hyperlink{#1back}{#2}}\hypertarget{#1}{{\begin{centering}$#1$\end{centering}}}\end{frame}}
\newcommand{\answer}[1]{\begin{itemize}\item #1\end{itemize}}


\newcommand{\jumpto}[2]{\hypertarget{#1back}{\hyperlink{#1}{#2}}}
\newcommand{\backto}[2]{\hypertarget{#1}{\hyperlink{#1back}{#2}}}


\title[Balassa--Samuelson]% (optional, use only with long paper titles)
{A Spatial Explanation for the Balassa--Samuelson Effect}


\author[Kar\'adi and Koren] % (optional, use only with lots of authors)
{P\'eter Kar\'adi (NYU)\\
Mikl\'os Koren (Princeton)}
% - Give the names in the same order as the appear in the paper.
% - Use the \inst{?} command only if the authors have different
%   affiliation.


\date % (optional, should be abbreviation of conference name)
{}
% - Either use conference name or its abbreviation.
% - Not really informative to the audience, more for people (including
%   yourself) who are reading the slides online

%\subject{Theoretical Computer Science}
% This is only inserted into the PDF information catalog. Can be left
% out.



% If you have a file called "university-logo-filename.xxx", where xxx
% is a graphic format that can be processed by latex or pdflatex,
% resp., then you can add a logo as follows:

\pgfdeclareimage[height=0.5cm]{university-logo}{frblogo}
%\logo{\pgfuseimage{university-logo}}



% Delete this, if you do not want the table of contents to pop up at
% the beginning of each subsection:
\AtBeginSection[]
{
  \begin{frame}[plain]
    \frametitle{\hfill\color{red}\insertsection}
    \addtocounter{framenumber}{-1}
    %\tableofcontents[currentsection,currentsubsection]
  \end{frame}
}


% If you wish to uncover everything in a step-wise fashion, uncomment
% the following command:

%\beamerdefaultoverlayspecification{<+->}

\setbeamertemplate{navigation symbols}{}
\setbeamertemplate{footline}{{}\hfill\insertframenumber--\oradisplay{percek}}

\begin{document}

\begin{frame}[plain]
  \titlepage
    \addtocounter{framenumber}{-1}
\end{frame}


\section{Introduction}
\subsection{Motivation}
\begin{frame}\frametitle{The Balassa--Samuelson effect}

\begin{itemize}
    \item Rich countries are more expensive than poor ones.
    \item In Penn World Tables,
    \[
    \ln P = 0.25\ln Y + e.
    \]
    %% The reason I put in numbers here is that I want to be able to do back of the env calcs.
    \item This is caused by differences in non-tradable prices, as tradable prices are equalized.
\end{itemize}
\end{frame}

%% show 2 hsieh klenow graphs

\begin{frame}\frametitle{The Balassa--Samuelson explanation}

\begin{itemize}
    \item Productivity differences in non-tradables are smaller.
    \item Prices in country $i$
    \begin{align*}
    p_{T,i} &= w_i/A_{T,i}\\
    p_{NT,i} &= w_i/A_{NT,i}
    \end{align*}
    \item and relative prices
    \[
    \frac{p_{NT,i}}{p_{T,i}} = \frac{A_{T,i}}{A_{NT,i}}
    \]
    are highly correlated with $Y(A_{T,i},A_{NT,i})$.
    \item But why is technical progress slower for non-tradables?
\end{itemize}
\end{frame}


\begin{frame}\frametitle{The Kravis--Lipsey--Bhagwati explanation}
\begin{itemize}
    \item Non-tradables are more intensive users of the non-reproducible factor (labor).
    \item This raises their price with capital accumulation.
    \item But the difference in labor intensity is small (Herrendorf and Valentinyi, 2007).
    \item ***Also, is it just a convenient correlation that non-tradables use more non-reproducible factor?
    \item ***Hard to do counterfactuals in this case.
    \item We argue that it's not.
\end{itemize}
\end{frame}

%% B-S is relatively recent


\subsection{Basic idea of the model}
\begin{frame}\frametitle{Basic idea}
\begin{itemize}
    \item Tradable sectors locate to where land is cheap.
    \item Non-tradable sectors have to locate near consumers in big cities.
    \item They compete with housing for scarce urban land.
    \item Urban land becomes more and more scarce with development.
    \item Raising the relative price of non-tradables.
\end{itemize}
\end{frame}

\begin{frame}\frametitle{Key ingredients}
\begin{itemize}
    \item Land is scarce and has a non-negligible share in production.
    \item Demand for residential land increases with development.
    \item Tradable sectors move out of cities.
\end{itemize}
\end{frame}

%\section{Outline}
\section{Key ingredients}
\begin{frame}\frametitle{Land is scarce}
\begin{itemize}
    \item Population density of the Earth is 42$/\text{km}^2$, so land is abundant.
    \pause
    \item However, the average person lives in an area with a population density of 7,300$/\text{km}^2$ (LandScan 2005),
    so \emph{land close to consumers} is scarce.
\end{itemize}
\end{frame}

\begin{frame}\frametitle{Land share}
\begin{itemize}
\item Decreased from 25\% in mid 1700s to around 5\% now --- mainly
because of the shrinking share of agriculture in GDP (Clark, 2007)
\item Sector income shares in various industries in the US (Herrendorf and Valentinyi, 2007)
\begin{center}
\begin{tabular}{l|c|ccc}
\hline\hline
Industry    & Capital   &   Land    & Structures    & Equipment \\ \hline
GDP         &   0.32    &   0.05    & 0.13          & 0.14      \\ \hline
Agriculture & 0.43      &   0.18    & 0.10          & 0.15      \\
Manufacturing & 0.31    &   0.03    & 0.08          & 0.20      \\
Services    & 0.32      & 0.05      & 0.15          & 0.12      \\ \hline\hline
\end{tabular}
\end{center}
\end{itemize}
\end{frame}

\widefigure{clark-landrents}{Agricultural and urban land rents in England (Clark, 2007)}


\begin{frame}\frametitle{Demand for residential land increases with development}
\begin{itemize}
    \item Between 1976 and 1992, U.S.~GDP per capita rose by 37\%.
    \item During this same period, residential land per capita increased by 25\%.
    (Burchfield, Overman, Puga and Turner, 2006; Overman, Puga and Turner, 2007)
    \item Between 1950 and 2000, the share of land in the value of a home increased from 10\% to 36\%. (Davis and Heathcote, 2007)
    \item During the same period, the price of residential land increased more than nine-fold.
\end{itemize}
\end{frame}

\widefigure{dh-shares}{The share of land in home value (Davis and Heathcote, 2007)}

%% skip this in the interest of time

\begin{frame}\frametitle{Income and the demand for housing}
\begin{table}[h!]
\center 
\begin{tabular}{l|cc}
  \hline\hline
  Explanatory & \multicolumn{2}{c}{Dependent variable} \\
  variables & Land value & Number of rooms \\ 
            & (log)      & per capita (log) \\
  \hline
  Income (log)   & \textbf{2.77}   & \textbf{0.26}\\
                 & (0.67)          & (0.08)\\
  Population (log)    & 0.13            &  \textbf{-0.07}\\
                 & (0.18)          & (0.01) \\ \hline
  $R^2$          & 0.42            & 0.26\\
  No. of obs.    & 46              & 3219\\ \hline\hline
\end{tabular}
\end{table}
\end{frame}

\begin{frame}\frametitle{Tradable sectors move out of cities}
\begin{itemize}
    \item Burchfield, Overman, Puga and Turner (2006): commercial land is more scattered than residential land, more so in 1992 than in 1976.
    \item Holmes and Stevens (2004): in 1997 manufacturing is underrepresented in large cities.
    \item Desmet and Fafchamps (2006): manufacturing deconcentrated between 1970 and 2000.
\end{itemize}
\end{frame}

\widefigure{LQ_map}{Locational quotient of tradable sectors}
\widefigure{LQ_scatter}{Tradables stay away from dense counties}

\section{Model}
\subsection{Basic structure}
\begin{frame}\frametitle{Basic structure}
\begin{itemize}
    \item There are three industries, manufacturing ($m$), services ($s$), and housing ($h$).
    \item Technology is identical in the two productive sectors.
    \item Goods are costly to transport.
\end{itemize}
\end{frame}

\begin{frame}\frametitle{Spatial structure}
\begin{itemize}
    \item We model countries as monocentric cities (von Th\"unen).
	\item All market exchange takes place in a central business district (CBD).
	\item CBD is a point in the plain.
	\item Residents, manufacturing and service establishments can choose their location freely in the plain.
	\item Location is indexed by distance to the CBD, $z$.
\end{itemize}
\end{frame}

%% graph of CBD

\begin{frame}\frametitle{Technology}
\begin{itemize}
    \item Land is the only factor of production. (We add labor later.)
	\item Production functions:
\begin{align*}
m&=A_ml_m\\
s&=A_sl_s\\
h&= A_hl_h
\end{align*}
\end{itemize}
\end{frame}

\begin{frame}\frametitle{Tastes}
\begin{itemize}
    \item Consumers have a homothetic utility function over $m$, $s$ and $h$.
    \item With indirect utility function
    \[
    u(I,p_m,p_s,p_h) = \frac{I}{P(p_m,p_s,p_h)}.
    \]
\end{itemize}
\end{frame}


\begin{frame}\frametitle{Transport costs}
\begin{itemize}
    \item Goods are shipped to the CBD.
    \item Both manufacturing and services have iceberg transport cost.
    \item One good $i$ shipped from location $z$ melts to
    \[
    1-\tau_iz.
    \]
    \item Services are less tradable:
    \[
    \tau_s>\tau_m.
    \]
\end{itemize}
\end{frame}

\begin{frame}\frametitle{Commuting costs}
\begin{itemize}
    \item People go to the CBD to shop.
    \item Commuting distance $z$ costs $\tau_h z$ fraction of the consumption bundle.
    \item So that indirect utility is
    \[
    u(I,p_m,p_s,p_h) = \frac{(1-\tau_h z)I}{P(p_m,p_s,p_h)}.
    \]
    \item Commuting is the costliest of all,
    \[
    \tau_h>\tau_s>\tau_m.
    \]
\end{itemize}
\end{frame}


\begin{frame}\frametitle{Equilibrium}
\begin{itemize}
    \item Firms maximize profits and choose location optimally.
    \item Households maximize utility and choose residence optimally.
    \item Manufacturing and service markets clear at the CBD.
\end{itemize}
\end{frame}

\begin{frame}\frametitle{Profit maximization}
\begin{itemize}
    \item Land rent at location $z$: $r(z)$.
    \item Profits for industry $i$ at location $z$:
    \[
    (1-\tau_i z) p_i A_i l_i(z) - r(z)l_i(z).
    \]
    \item Optimum requires
    \[
    (1-\tau_i z) p_i A_i \le r(z),
    \]
    with $=$ if industry $i$ produces at location $z$.
\end{itemize}
\end{frame}

\begin{frame}\frametitle{The bid rent curve}
\begin{itemize}
    \item Define a bid rent curve:
    \[
    R_i(z) = p_iA_i(1-\tau_i z).
    \]
    \item Profit maximization requires
    \[
    r(z)\ge R_i(z)
    \]
    \item Industry $i$ produces at location $z$ only if $=$.
    \item Rent $r(z)$ is the upper envelope of the bid rent curves.
\end{itemize}
\end{frame}

%% graph of two bid rent curves

\begin{frame}\frametitle{The bid rent curve of households}
\begin{itemize}
    \item Housing at $z$ costs $r(z)/A_h$.
    \item Other two prices do not depend on residence.
    \item To achieve utility $u$ at location $z$,
    \[
    uP[p_m,p_s,R_h(z)/A_h] = (1-\tau_h z) I.
    \]
\end{itemize}
\end{frame}



\begin{frame}\frametitle{Equilibrium city structure}
\begin{itemize}
    \item Residents live closest to CBD, $\in[0,z_1]$.
    \item Followed by a ring of service establishments, $\in(z_1,z_2]$.
    \item Followed by a ring of manufacturing plants, $\in(z_1,z_3]$.
    \item City boundary is $z_3 = 1/\tau_m$.
\end{itemize}
\end{frame}

%% graph of three bid rent curves

%% graph of concentric circles

\begin{frame}\frametitle{Aggregate supplies}
\begin{itemize}
    \item All land between $z_1$ and $z_2$ is allocated to services.
    \item A fraction $\tau_s z$ gets lost in transit.
    \item Overall supply:
    \begin{align*}
    s &= \int_{z_1}^{z_2}2\pi z (1-\tau_s z) dz = s(\underset{-}{z_1},\underset{+}{z_2})\\
    m &= \int_{z_2}^{z_3}2\pi z (1-\tau_m z) dz = m(\underset{-}{z_2},z_3)
    \end{align*}
\end{itemize}
\end{frame}

\begin{frame}\frametitle{The relative price of services}
\begin{itemize}
    \item At the manufacturing--service boundary $z_2$,
    \[
    p_mA_m(1-\tau_m z_2) = p_sA_s(1-\tau_s z).
    \]
    \item The relative price of services
    \[
    \frac{p_s}{p_m} = \frac{A_m}{A_s}\frac{1-\tau_m z_2}{1-\tau_s z_2}.
    \]
    \item This increases if $A_m/A_s$ does or if $z_2$ does (because $\tau_m<\tau_s$).
\end{itemize}
\end{frame}


\begin{frame}\frametitle{Comparative statics}
\begin{itemize}
    \item
\end{itemize}
\end{frame}


\begin{frame}\frametitle{Propositions}
\begin{block}{Balanced growth}
Productivity growth does not change the relative price of services if
    \begin{enumerate}
      \item housing productivity grows at the same rate,
      \item \emph{or} demand for housing is Cobb--Douglas.
    \end{enumerate}
\end{block}
\begin{block}{Balassa--Samuelson and the sprawl}
Service prices rise with development if and only if residential land increases with development.
\end{block}
\end{frame}

%% explain C-D

%% explain sprawl



\begin{frame}\frametitle{Complementary housing}
\begin{itemize}
    \item Balassa--Samuelson requires an increase in residential land.
    \item Housing is complementary with $m$ and $s$ (Leontief).
    \item Productivity growth in housing is slower (zero).
\end{itemize}
\end{frame}


\section{Numerical exercise}
\section{Empirics}

\begin{frame}\frametitle{}

\begin{itemize}[<+->]
    \item
    \begin{enumerate}[<+->]
        \item
    \end{enumerate}
    \item
\end{itemize}
\end{frame}




\addtocounter{percek}{9}

\subsection{What we do}

\end{document}
