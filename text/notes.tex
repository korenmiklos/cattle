% ----------------------------------------------------------------
% AMS-LaTeX Paper ************************************************
% **** -----------------------------------------------------------
\documentclass{amsart}
\usepackage{graphicx}
% ----------------------------------------------------------------
\vfuzz2pt % Don't report over-full v-boxes if over-edge is small
\hfuzz2pt % Don't report over-full h-boxes if over-edge is small
% THEOREMS -------------------------------------------------------
\newtheorem{thm}{Theorem}[section]
\newtheorem{cor}[thm]{Corollary}
\newtheorem{lem}[thm]{Lemma}
\newtheorem{prop}[thm]{Proposition}
\theoremstyle{definition}
\newtheorem{defn}[thm]{Definition}
\theoremstyle{remark}
\newtheorem{rem}[thm]{Remark}
\numberwithin{equation}{section}
% MATH -----------------------------------------------------------
\newcommand{\norm}[1]{\left\Vert#1\right\Vert}
\newcommand{\abs}[1]{\left\vert#1\right\vert}
\newcommand{\set}[1]{\left\{#1\right\}}
\newcommand{\Real}{\mathbb R}
\newcommand{\eps}{\varepsilon}
\newcommand{\To}{\longrightarrow}
\newcommand{\BX}{\mathbf{B}(X)}
\newcommand{\A}{\mathcal{A}}
% ----------------------------------------------------------------
\begin{document}

\title{Notes}%
\author{Koren}%

%\date{}%
%\dedicatory{}%
%\commby{}%
% ----------------------------------------------------------------
% ----------------------------------------------------------------
\section{Derivation of Stage 3}
In the Stage 3 equilibrium, we have 5 final-good supplies (2 housing, 2 services, 1 manufacturing), 6 demands (manufacturing, services, housing in each locations),

\begin{enumerate}
  \item Supply of housing in village ($h_1,\lambda_1$)
  \item Supply of housing in city ($h_2,\lambda_2$)
\end{enumerate}

We collapse these into 4 equations and 4 unknowns by substituting in simple one-to-one relationships. First, we write the excess supply of manufacturing in the village as
\[
m_2(\lambda_1) = AF(N_1/L_1,1-\lambda_1) - \lambda_1/H.
\]
Similarly, the supply of services in the city is
\[
s_2(\lambda_2) = AF(N_2/L_2,1-\lambda_2).
\]
The rent in the village is
\[
r_1 = MPL_1(\lambda_1) = AF_L(N_1/L_1,1-\lambda_1).
\]
We then have a system of 4 unknowns, $\lambda_1$, $\lambda_2$, $r_2$ and $p_2$, and 4 equations:
\begin{align}
r_2(1-\alpha)L_2 &= \alpha r_1(\lambda_1) L_1 - m_2(\lambda_1) \tag{budget constraint}\\
r_2              &= p_2 MPL_2(\lambda_2) \tag{value marginal product}\\
\frac{m_2(\lambda_1)}{s_2(\lambda_2)} &= \frac{\gamma}{1-\gamma}p_2 \tag{relative demand of services}\\
m_2(\lambda_1)^\gamma s_2(\lambda_2)^{1-\gamma} &= \lambda_2 L_2/H \tag{relative demand of housing}
\end{align}
We combine BC, VMP and RDS into (check if it's correct)
\begin{equation}
\frac{\alpha}{1-\alpha}\frac{MPL_1(\lambda_1)}{m_2(\lambda_1)/L_1} - \frac{1}{1-\alpha} = \frac{1-\gamma}{\gamma} \frac{MPL_2(\lambda_2)}{s_2(\lambda_2)/L_2}.
\end{equation}
This schedule implies a positive relationship between $\lambda_1$ and $\lambda_2$. A higher share of housing in the village reduces the supply of manufactured goods, thereby lowering the price of services in the city. This reduces the value marginal product of land in the city, inducing a move towards housing. The higher rent in the village also increases the rent transfers to the city, which further increases housing demand there.

The other condition is RDH,
\begin{equation}
m_2(\lambda_1)^\gamma s_2(\lambda_2)^{1-\gamma} = \lambda_2 L_2/H,
\end{equation}
which implies a negative relationship between $\lambda_1$ and $\lambda_2$. Higher housing share in the village implies less manufactured good, which have to be compensated with more services (which would lead to less housing share in the city) and/or less housing so that city consumption is optimal.

The intersection of these two schedules (if exists) is unique.
\subsection{To do}
\begin{enumerate}
  \item Verify existence. Write out parameter conditions.
  \item Verify that $p_2>1$, which is the condition for complete specialization to be an equilibrium. Write out parameter restrictions.
  \item (?) Show that incomplete specialization or wrong-sided specialization are not equilibriua.
  \item Conduct comparative statics wrt $A$ and $\alpha$.
\end{enumerate}
\section{Spatial version}
We use the von Th\"unen monocentric city model. There is a central business district (CBD), which is a point in the plain. Businesses and residences can locate near this CBD. The CBD serves as a marketplace: all goods and services are exchanged there.

There are two produced goods, and housing. Both goods are produced using land only (introducing labor is simple) and they are costly to transport to the CBD. Manufactured goods are assumed to have lower transport costs than services.

Consumers consume manufactured goods, services and housing (in which the single input is land). They do not work, their only income $I$ is from a share of total city rents. They commute to the CBD to buy the two goods.

All transport and commuting cost is of the iceberg nature. For a manufacturing plant $z$ miles from the center, the transport cost is $\tau_m z$ fraction of manufactured goods. For a service establishment, transport cost is $\tau_s z$. Commuting $z$ miles costs $\tau_h z$ units of the consumption bundle (manufacturing, service, housing). (One can think of this as time lost from consuming / enjoying the house.)

Technology in the two sectors is identical,
\begin{align*}
m&=Al,\\
s&=Al,
\end{align*}
where $l$ is the amount of land allocated to production, and $A$ is the yield per land. Technical progress will be captured as an increase in $A$.

The profit of a firm in industry $i$ $z$ miles from the center is 
\[
\pi_i(z,l) = p_i(1-\tau_i z)Al - r_zl,
\]
where $r_z$ is the land rent in location $z$. 

Profit maximization leads to
\[
p_i(1-\tau_i z)A \le r_z,
\]
with equality when there is positive production. This pins down industry $i$'s bid rent function,
\[
R_i(z,p_i) = p_i(1-\tau_i z)A.
\]
This is the maximum rent the industry is willing to pay at location $z$. It is linearly decreasing in $z$ (GRAPHS). Note that the bid rent function is steeper for services because $\tau_s>\tau_m$.


There is a continuum of consumers with a mass $N$. Utility is homothetic
\[
u(m,s,h),
\]
so that indirect utility for a household $z$ miles from the center can be written as
\[
\frac{I(1-\tau_hz)}{P(p_m,p_s,r_z)}.
\]
Here $p_m$ is the (one) price of manufacturing, $p_s$ is the price of services, and $r_z$ is the land rent in location $z$.

Suppose the level of utility is $u$ in all locations (this is equalized across locations, but the level $u$ is endogeoues). Then the bid rent function of households is such that
\[
P(p_m,p_s,R_z) = \frac Iu (1-\tau_h z).
\]

For given income and good prices, the optimal location choice of households determines their bid rent function, $R_h(z,I,p_m,p_s)$. Because the indirect utility function is homogeneous of degree zero in $I$, $p_m$, $p_s$ and $r_z$, the bid rent function is homogeneous of degree one in $I$, $p_m$ and $p_s$.
\subsection{Equilibrium}
In equilibrium, firms maximize profits (and hence choose location optimally), consumers maximize utility (and hence choose location optimally), the global markets for the two goods clear, as does the local market for each plot of land. (Note that there will always be unused land far-enough from the center, because it is not feasible to commute/transport to long distances.)

We conjecture an equilibrium of the following spatial structure. Residents live closest to the center, followed by a ring of service establishments, and by a ring of manufacturing belt. (It probably suffices to set $\tau_h>\tau_s>\tau_m$.) Let $z_1$ denote the boundary of the service sector, and $z_2>z_1$ the boundary of the manufacturing sector. The city boundary is denoted by $z_3 = 1/\tau_m$. It is not feasible to transport manufactured goods from outside this boundary. By choice of distance units, we normalize $z_3=1$, which is equivalent to $\tau_m=1$.

The supply of services is $A$ times the amount of land used in services, $l_s = (z_1^2-z_1^2)\pi$:
\[
s = A(z_2^2-z_1^2)\pi.
\]
Similarly, the supply of manufacturing is
\[
m = A(1-z_2^2)\pi.
\]

Suppose a mass of consumers consume $h(z)$ of housing at location $z$. In other words, $n(z) = 1/h(z)$ is the population density. 

The total population of the city is the integral of the density between $0$ and $z_1$,
\[
N = \int_{0}^{z_1}2\pi z n(z) dz.
\]

Arbitrage conditions for the boundaries
\begin{align}
R_h(z_1,I,p_s,p_m) &= R_s(z_1,p_s)\\
R_m(z_2,p_m) &= R_s(z_1,p_s)
\end{align}
\subsection{Comments}
\begin{enumerate}
  \item It is easy to show that either Cobb-Douglas utility or symmetric TFP increase in housing kills our effect. We need slower TFP increase (0) \emph{and} complementary housing (Leontief) to make this work.
\end{enumerate}
\subsection{Leontief demand}
Suppose utility is
\[
u(m,s,h) = \min\{m^\gamma s^{1-\gamma}, h/H\}.
\]
The price index is then
\[
P(p_m,p_s,r_z) = c p_m^\gamma p_s^{1-\gamma} + H r_z,
\]
where $c$ is some fancy constant that I always forget. The residential bid rent function is 
\[
R_h(z) = \frac{I}{uH}(1-\tau_h z) -\frac cH p_m^\gamma p_s^{1-\gamma}.
\]
Now all the bid rent functions are linear in $z$, which should make the determination of the cutoffs simpler.

Housing demand is the derivative of the expenditure function with respect to $r_z$,
\[
h(z) = u H,
\]
independent of location. This way, population density is constant at $n(z) = 1/(uH)$, which pins down $u$ as a function of $z_1$
\[
u = \frac{\pi}{N H}z_1^2.
\]
The residential area is $\pi z_1^2$, so the amount of land per household is $\pi z_1^2/N$, each unit measure giving a utility of $1/H$.

\subsection{Constructing the equilibrium}
We pick the manufactured good as the  numeraire, $p_m=1$. Writing out the three bid rent functions again,
\begin{align*}
R_s(z) &= Ap_s(1-\tau_s z)\\
R_h(z) &= \frac{IN}{\pi z_1}(1-\tau_h z) -\frac {c}{H} p_s^{1-\gamma}\\
R_m(z) &= A(1-z)
\end{align*}
We need to find the equilibrium $z_1$ and $z_2$.We will need two conditions. 

The price of services can be pinned down from the service / manufacturing arbitrage,
\[
Ap_s(1-\tau_s z_2) = A(1-z_2),
\]
\[
p_s = \frac{1-z_2}{1-\tau_s z_2}.
\]
Because $\tau_s>1$, $p_s$ is increasing in $z_2$.

The relative demand to manufacturing is increasing in $p_s$ and hence $z_2$,
\[
\frac{m}{s} = \frac{\gamma}{1-\gamma}p_s = \frac{\gamma}{1-\gamma}\frac{1-z_2}{1-\tau_s z_2}.
\]
The relative supply of manufacturing is simply
\[
\frac{m}{s} = \frac{A(1-z_2^2)\pi}{A(z_2^2-z_1^2)\pi} = \frac{1-z_2^2}{z_2^2-z_1^2}.
\]
This is decreasing in $z_2$ and increasing in $z_1$.

Equating relative supply to relative demand creates a schedule between $z_1$ and $z_2$, with a positive slope. This is condition one.

The second relative demand condition does not depend on prices because of the Leontief assumption,
\[
m(z)^\gamma s(z)^{1-\gamma} = h(z)/H.
\]
Notice that because this holds for all households in all locations $z$, and because $m/s$ does not vary across locations (the law of one price holds for services and manufacturing), the condition also holds in the aggregate,
\[
m^\gamma s^{1-\gamma} = h/H.
\]
We can then substitute in the aggregate supply functions to get the second condition for $z_1$ and $z_2$,
\[
A\pi (1-z_2^2)^\gamma(z_2^2-z_1^2)^{1-\gamma} = \pi z_1^2/H.
\]
This is condition two. Note that if there was a parallel TFP increase in housing, $z_1$ and $z_2$ would not be affected by TFP.

\end{document}
% ----------------------------------------------------------------
