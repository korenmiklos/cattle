\documentclass[10pt]{article}

\usepackage{amsmath,amssymb}
\usepackage{multirow}
\usepackage{graphicx}

\addtolength{\hoffset}{-1cm} \addtolength{\textwidth}{2cm}


\begin{document}
\title{Spatial model with location specific labor\thanks{Work in progress}}
\maketitle

\begin{enumerate}
\item New assumption: Workers have to work where they live ($z$), so the new variable is $w(z)$, and the equilibrium (Def. 1.4) requires that labor market clears at each location.
\item Variables:
\begin{table}[h!]
\center
\begin{tabular}{ll}
$l_i(z)$ & land used for production, $i=m,s$ \\
$l_h(z)$ & land used for living \\
$n(z)$ & labor employed\\
$m(z)$ & manufacturing product bought at $z$ \\
$s(z)$ & services bought at $z$ \\
$h(z)$ & housing demanded at $z$ \\
$r(z)$ & rents \\
$w(z)$ & wages \\
$p_m$ & price of the manufacturing product at the CBD \\
$p_s$ & price of services at the CBD \\
$p_h(z)$ & price of housing at $z$ bought at the CBD \\
$u$ & utility of the consumers\\
$z_1$ & boundary between manufacturing and services production \\
$z_2$ & production boundary \\
\end{tabular}
\end{table}

\item \textbf{Firms}
Firms take the prices ($p_m,p_s$) and wages ($w(z)$) and rents ($r(z)$) as given and at each location $z$ choose what to produce ($m,s$), how much labor to employ ($n^d(z)$) and how much land to use ($l^d(z)$). In equilibrium land goes to the highest bidder, and firms have zero profits, so the first order condition for a profit maximum is
\begin{equation*}
p_iD(\tau_iz)\leq \frac{r(z)^\beta w(z)^{1-\beta}}{A_i\beta^\beta(1-\beta)^{1-\beta}},
\end{equation*}
from which for $i=m,s$ we get
\begin{equation}\label{eq:bidrent:industry}
R_i(z,p_i,w(z)) = \beta (1-\beta)^{1/\beta-1} p_i^{1/\beta} A_i^{1/\beta} D(\tau_i z)^{1/\beta}w(z)^{1-1/\beta}.
\end{equation}
As under the same wages ($w(z)$) and $\beta$ and $A_i$, it is still only the transportation costs ($\tau_i$) and the prices ($p_i$) which will determine the difference in rent-bids of the industries, the spatial structure will presumably stay the same: services closer to the center, manufacturing further out, with $z_1$ in between.

From the first order conditions we also know that $\forall z$
\begin{equation}
\frac{l(z)}{n(z)}=\frac{\beta}{1-\beta}\frac{w(z)}{r(z)},
\end{equation}
and
\begin{equation}
r(z)=p_iD(\tau_i z)A_i\beta\left(\frac{l(z)}{n(z)}\right)^{\beta-1}.
\end{equation}

Zero profit condition in the housing sector means that
\begin{equation}
p_h(z)=\frac{r(z)}{A_h}
\end{equation}

\item \textbf{Consumers}
The consumers maximize $u(m,s,h)$ subject to $\forall z$
\begin{equation*}
p_mm(z)+p_ss(z)+p_h(z)h(z)=(w(z)+T)D(\tau_hz).
\end{equation*}
The free mobility assumption of consumers ensure that $\forall z$
\begin{equation}
u=\frac{D(\tau_hz)(w(z)+T)}{P\left[\phi(p_s,p_m),r(z)/A_h\right]},
\end{equation}
where we used the fact that rents paid by the consumers at location $z$ needs to be the same as the rents paid by the firms at the same location. This equation here is not the rent bid function of the consumers, but rather an equation determining equilibrium $w(z)$ as a function of $r(z)$ and other variables ($u, p_s, p_m,T$):
\begin{equation*}
w(z)=\frac{uP\left[\phi(p_s,p_m),r(z)/A_h\right]}{D(\tau_hz)}-T.
\end{equation*}
Assuming decreasing equilibrium rents $r(z)$ this is decreasing with distance. Note: there can be a $\bar{z}$ such that $w(\bar{z})=0$ after which individuals will not want to work just live from their rent income $T$.

I will need solutions for $m(z),s(z),h(z)$ $\forall z$, especially I am not sure how we can go around having an expression for $h(z)$ (see later).

Producing $h(z)$ housing needs $l_h(z)=h(z)/A_h$ land. The amount of workers at $z$ is given by
\begin{equation*}
n(z)=\frac{l_h(z)}{h(z)/A_h}.
\end{equation*}
At each location $z$ we have unit land available, so $l_h(z)=1-l(z)$, from which
\begin{equation}
n(z)=\frac{1-l(z)}{h(z)/A_h}
\end{equation}

\item \textbf{Market clearing}
Output at location ($z$) is given by
\begin{equation*}
y_i(z)=(1-\beta)^{1/\beta-1}p_i^{1/\beta-1}A_i^{1/\beta}D(\tau_iz)^{1/\beta}w(z)^{1/\beta-1}l(z),
\end{equation*}
implying that the supplies of the two industries are
\begin{equation*}
s=(1-\beta)^{1/\beta-1}p_s^{1/\beta-1}A_s^{1/\beta}\int_0^{z_1}D(\tau_sz)^{1/\beta}w(z)^{1/\beta-1}l(z)dz,
\end{equation*}
\begin{equation*}
m=(1-\beta)^{1/\beta-1}p_m^{1/\beta-1}A_m^{1/\beta}\int_{z_1}^{z_2}D(\tau_mz)^{1/\beta}w(z)^{1/\beta-1}l(z)dz,
\end{equation*}

The location arbitrage at $z_1$ implies $R_s(z_1)=R_m(z_2)$ and as a result of the free mobility equation, we also know that the wage $w(z_1)$ does not depend on which industry is operating on the location. So, similarly to previous case, we have that
\begin{equation}
\frac{p_s}{p_m}=\left[\frac{D(\tau_mz_2)}{D(\tau_sz_2)}\right]^{1/\beta}.
\end{equation}
Relative prices determine relative demand for manufacturing ($m$) and services ($s$).

The boundaries $z_1$ and $z_2$ are determined such that product markets clear. The city boundary $z_2$ is the smaller of $\bar{z}$ and $\hat{z}$ such that $D(\tau_m\hat{z})=0$.
\end{enumerate}

\subsection{Cobb--Douglas, exponential case}
Write the firms' bid rent curve as
\begin{equation*}
R_i(z,p_i,w(z)) = \beta (1-\beta)^{1/\beta-1} p_i^{1/\beta} A_i^{1/\beta} D(\tau_i z)^{1/\beta}w(z)^{1-1/\beta}.
\end{equation*}
This is decreasing in wages.

Similary, we can write the bid rent curve of households as a function of the wage schedule.
\[
R_h(z,P,w(z)) \propto D(\tau_h z)^{1/\gamma}w(z)^{1/\gamma},
\]
where $\gamma$ is the CD share of housing in consumption expenditure. This increases in wages. This assumes $T=0$ (absentee landlord), but is also valid for rents redistributed in proportion to wage income.

In mixed-used equilibrium it has to be the case that over some interval $Z$, $R_i(z)=R_h(z)$. For this to be true, the wage schedule has to satisfy a functional equation,
\[
w(z) \propto \left[\frac{D(\tau_hz)^{1/\gamma}}{D(\tau_iz)^{1/\beta}}\right]^{1/(1-1/\beta-1/\gamma)}.
\]
With exponential transport costs this simplifies to
\[
\ln w(z) = c - z\frac{\tau_i/\beta-\tau_h/\gamma}{1/\beta+1/\gamma-1},
\]
which decreases in $z$ if and only if $\tau_i/\beta > \tau_h/\gamma$. (This, by the way, is the condition that the service bid rent curve is steeper than the residential. Intuitively, if residents wanted to live in the center so much, they would be happy with a lower wage.)

The rent gradient is
\[
\ln r(z) = c - z\left[\frac{1/\gamma}{1/\beta+1/\gamma-1}\tau_i/\beta+
\frac{1/\beta-1}{1/\beta+1/\gamma-1}\tau_h/\gamma\right],
\]
always decreasing in $z$. If $\gamma\to 0$, the first term converges to $\tau_i/\beta$, the rent gradient without LSL.

The relative wage is
\[
\ln w(z) - \ln r(z) = c + z\left[\frac{1/\gamma-1}{1/\beta+1/\gamma-1}\tau_i/\beta+
\frac{1/\beta}{1/\beta+1/\gamma-1}\tau_h/\gamma\right],
\]
increasing in $z$. Because of this, commercial land per employment increases with $z$.

Also, by the Cobb--Douglas demand for housing, residential land per employment also increases with $z$. These two imply that population density declines with $z$. Notice that we don't even need $\tau_h$, we can zero it out for simplicity. Cities are more dense because service establishments like to be close to markets, bid up the rent, and increase the relative demand for labor.

\subsection{Equilibrium land use at location $z$}
Assuming $\tau_h=0$ implies an especially nice constant commercial/residential land share. Under this assumption the C-D housing demand implies
\begin{equation*}
h(z)=\gamma A_h\frac{w(z)}{r(z)},
\end{equation*}
so $h(z)$ increases in the same rate as $w(z)/r(z)$. \footnote{Under positive $\tau_h$ it is this equation, that would not be this easy.}

The residential land per population depends on $h(z)$, so we will have
\begin{equation*}
\frac{l_h(z)}{n(z)}=\frac{h(z)}{A_h}=\gamma\frac{w(z)}{r(z)},
\end{equation*}
and the relative factor prices also determine the equilibrium development of commercial land per population
\begin{equation*}
\frac{l(z)}{n(z)}=\frac{\beta}{1-\beta}\frac{w(z)}{r(z)}. 
\end{equation*}
Adding the two equation together, using the land constraint of $l_h(z)+l(z)=1$ and solving for $n(z)$ implies
\begin{equation}
n(z)=\frac{1-\beta}{\gamma(1-\beta)+\beta}\frac{r(z)}{w(z)},
\end{equation}
implying that the population density ($n(z)$) decreasing in the same rate as the relative $w(z)/r(z)$ rate. 

Using the equation for commercial land we get that 
\begin{equation}
l(z)=\frac{\beta}{\gamma(1-\beta)+\beta}
\end{equation}
constant. Commercial land used is given by $l_h(z)=1-l(z)=\frac{\gamma(1-\beta)}{(\gamma(1-\beta)+\beta)}$. 

*** This is going to be a problem, since services is no more ``exposed'' to residential land than manufacturing. Even though they have higher population density, everyone buys a smaller house because of the high city rents. This is the spatial equivalent of the C--D curse we had before. My guess is that if/when we go back to the Leontief case, people close to the CBD will have \emph{relatively} bigger homes (because they do not substitute out of housing one to one) and the center will be more residential. In that variant, however, the wage schedule is messy. 
\subsection{Solving for the equilibrium}
In order to obtain the equilibrium, we need to solve for the variables $p_m,p_s,z_1,s,m$ and the wage and rent functions ($w_s(z),w_m(z),r_s(z),r_m(z)$), where the subscripts refer to the industry offering the factor prices. The wage and rent functions are loglinear and to obtain a solution we need to solve for the constants for the (log) wage functions ($c^w_s,c^w_m$) and for the (log) rent functions ($c^r_s, c^r_m$). We can set the wages at the center as a numeraire $w_s(0)=1$ implying $c^w_s=0$. The facts that the rents and the wages offered at $z_1$ needs to be equal in equilibrium implies furthermore that
\begin{equation*}
w_m(z_1)=w_s(z_1) \Leftrightarrow c^w_m=-\frac{(\tau_s-\tau_m)/\beta}{1/\beta+1/\gamma-1}z_1,
\end{equation*}
and 
\begin{equation*}
r_m(z_1)=r_s(z_1) \Leftrightarrow c^r_m-c^r_s=\frac{(\tau_s-\tau_m)/(\gamma\beta)}{1/\beta+1/\gamma-1}{z_1}.
\end{equation*}
The value of $c_r^s$ is determined implicitly by the equation
\begin{equation*}
\int_0^\infty n(z)dz=N,
\end{equation*}
where $N$ is a parameter determining the overall population in the city. Note: under exponential transportation costs $z_2=\infty$, so the available land is also $\infty$, meaning that average population density is 0.



TODO: Integrate out to obtain the supply of services and manufacturing. Given the Cobb--Douglas assumption about housing, try the following alternative comparative statics: increase $\gamma$, increase everyone's income exogenously (in proportion to wages so as not to mess up the current spatial equilibrium).



\end{document} 