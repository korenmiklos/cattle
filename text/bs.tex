\documentclass[12pt]{article}
\usepackage{amsmath,amsthm}
\usepackage[latin1]{inputenc}
\usepackage[T1]{fontenc}
%\usepackage{minionpro}
\usepackage{array,graphicx}

\setlength{\topmargin}{-0.3in} \setlength{\textheight}{8.75in}
\setlength{\oddsidemargin}{0.0in} \setlength{\evensidemargin}{0.0in}
\setlength{\textwidth}{6.5in}
\def\labelenumi{\arabic{enumi}.}
\def\theenumi{\arabic{enumi}}
\def\labelenumii{(\alph{enumii})}
\def\theenumii{\alph{enumii}}
\def\p@enumii{\theenumi.}
\def\labelenumiii{\arabic{enumiii}.}
\def\theenumiii{\arabic{enumiii}}
\def\p@enumiii{(\theenumi)(\theenumii)}
\def\labelenumiv{\arabic{enumiv}.}
\def\theenumiv{\arabic{enumiv}}
\def\p@enumiv{\p@enumiii.\theenumiii}
\pagestyle{plain}
\pagestyle{plain} \setcounter{secnumdepth}{3}
\newcommand{\D}{\mathop{\mathrm{d\mathstrut}}\nolimits\!}
\newcommand{\dt}{\D t}
\newcommand{\dz}{\D z}
\newcommand{\E}{\mathop{\mathrm{E\mathstrut}}\nolimits}
\newcommand{\Var}{\mathop{\mathrm{Var\mathstrut}}\nolimits}
\newcommand{\sd}{\mathop{\mathrm{sd\mathstrut}}\nolimits}
\newcommand{\diag}{\mathop{\mathrm{diag\mathstrut}}\nolimits}
\newcommand{\Cov}{\mathop{\mathrm{Cov\mathstrut}}\nolimits}
\newcommand{\Corr}{\mathop{\mathrm{Corr\mathstrut}}\nolimits}
\newtheorem{definition}{Definition}
\newtheorem{proposition}{Proposition}
\newtheorem{conjecture}{Conjecture}
\newtheorem{moment}{Empirical regularity}
\newtheorem{insight}{Qualitative prediction}

\newcommand{\dofigure}[2]{\begin{figure}
\begin{centering}
\includegraphics[width=0.75\linewidth]{figures/#1}
  \caption{#2\label{fig:#1}}
\end{centering}
\end{figure}}

\begin{document}

\title{A Spatial Explanation for the Balassa--Samuelson Effect\thanks{\textsc{Preliminary.} We thank Jonathan Eaton, Nobu Kiyotaki, Esteban Rossi--Hansberg and seminar participants at Princeton for useful comments.}}
\author{P\'eter Kar\'adi\thanks{New York University. E-mail: peter.karadi@nyu.edu}~ and Mikl\'os Koren\thanks{Central European University and CEPR. E-mail: miklos@koren.hu.}}
\maketitle

\begin{abstract}
We propose a model of urbanization and development to explain why the price level is higher in rich countries. There are two sectors: manufacturing, which is freely tradable, and non-tradable services, which have to locate near customers in big cities. As countries develop, total factor productivity increases simultaneously in both sectors. However, because services compete with the population for scarce land, labor productivity will grow slower in services than in manufacturing. Services become more expensive, and the aggregate price level becomes higher. The model hence provides a theoretical foundation for the Balassa--Samuelson assumption that productivity growth is slower in the non-tradable sector than in the tradable sector. The key implications of the model are borne out in the data: while the positive correlation between income and the price level is very strong among countries where land is scarce (densely populated, highly urban countries), there is virtually no such correlation among rural countries.
\end{abstract}

\section{Introduction}
Rich countries are more expensive than poor ones. Balassa (1964) and Samuelson
(1964) suggested a simple and powerful explanation of this fact based on the
observation that the productivity growth of tradables, such as manufacturing,
tends to be faster than the productivity growth of non-tradables, such as
services. Assuming that labor is perfectly mobile across sectors,
firms need to pay the same wage in both sectors in equilibrium, and
these wages are pinned down by the international price of the
tradable good and the productivity in the tradable sector.
Relatively lower non-tradable sector productivity, thereby, will
imply higher non-tradable prices. Empirical observations support the
major propositions of the model about the importance of the
non-tradable sector as well as the development of the relative
productivities and their effects on the relative prices (see e.g.
Obstfeld-Rogoff, 1996)\footnote{Hsieh and Klenow (1997) recently
argued that low PPP investment rates in poor countries come from
their low relative productivity in producing investment and tradable
goods relative to non-tradable consumption services.} Though
empirical results are also consistent with the assumption that total factor
productivity growth does tend to favor the tradable sector, the
theory falls short of providing an explanation for it.

Rich cities are more expensive than poor ones. According to \texttt{bestplaces.net}, the overall cost of living is 54 percent higher in Boston than in Milwaukee. The two cities are of similar size (close to 600,000 people), but the per capita income in Boston (\$28,000) is about 56 percent higher than that in Milwaukee (\$18,000). Similarly to the price differences across countries, the biggest price differences are in nontradable sectors. Healthcare is 33 percent more expensive in Boston, whereas housing is about 200 percent more expensive.\footnote{All data retrieved from \texttt{http://bestplaces.net} on January 30, 2008.} While the Balassa--Samuelson explanation has the potential to explain price differences across cities, some simple demand considerations also jump to mind. Demand for services is higher in high-income cities, while supply is limited, which drives up service prices. A scarce factor in cities that may limit the supply of services is land.\footnote{Tang (2007) offers an income-based explanation for the positive correlation between price level and GDP per capita. He does not focus on cities and the spatial distribution of economic activity, however.}

We propose a model of development and urbanization that reconciles these two facts and connects these two types of explanations. There are two sectors: manufacturing, which is freely tradable, and non-tradable services, which have to locate near customers in big cities. As countries develop, total factor productivity increases simultaneously in both sectors. However, because services compete with the urban population for scarce land, labor productivity will grow slower in services than in manufacturing. Services become more expensive, and the aggregate price level becomes higher with development.

Our model rests on three key observations.

First, land is scarcer than one might first think. The overall population density of the earth is rather low, 43 people per square kilometer.  However, population is very clustered, so the average person lives in an area with a population density of 7,300 people per square kilometer.\footnote{We used high-resolution population density data from LandScan to calculate, for the average person on earth, the number of people living within the same 30 arc second by 30 arc second area. (This is basically a population-weighted population density.) By comparison, the population density of New York City is about 10,000 per square kilometer.} A related observation is that in 2005, around 50 percent of the world population lived in cities.\footnote{World Development Indicators.} This suggests that the scarcity of land, which has the potential to explain price differences across cities, can also be partly responsible for price differences across countries.

Second, whether a good is tradable or not has implications not only for cross-country trade but also for within-country trade, and hence the location of production within the country. Arguably, there are many technological differences between tradable sectors such as manufacturing, and non-tradable sectors such as services. One key difference is, however, that non-tradable goods or services have to be produced at the location of their consumption, whereas tradable goods and services can be produced far away from consumers, where land prices may be cheaper. Figure \ref{fig:ntcounties} shows a map of U.S.~counties, displaying the \emph{share} of employment in non-tradable sectors. Darker colors represent a higher share in non-tradable sectors. We see that non-tradable sectors locate closer to population centers (especially on the East and West Coasts and Florida) than tradable sectors.\footnote{This is confirmed more formally in Section \ref{empirics}.} The implication of this observation is that non-tradable prices are more sensitive to the price of land than tradable prices.

\dofigure{ntcounties}{The share of non-traded employment in U.S.~counties\newline
\small\emph{Source}: 2000 U.S.~Census, Summary File 3.}

Third, productivity growth in the housing sector is limited by the scarcity of land. Increased productivity in construction can lead to taller and higher-quality buildings, but they cannot substitute for land. Davis and Heathcote (2007) estimate that between 1975 and 2005, the share of land in the value of a home in the U.S.~has increased from 35 percent to 45 percent. During the same period, the price of residential land has more than tripled. This suggests that land and structures are complements and that the scarcity of residential land will become more severe with development.\footnote{Also see Section \ref{empirics} for more discussion of residential land prices and development.}
Together with the previous two observations we have that services, which locate close to their consumers in cities, face higher land rents with development. They become more labor intensive and exhibit lower labor productivity  for the same total factor productivity. Our model can hence be thought of as a \emph{microfoundation} for the assumption of Balassa and Samuelson that productivity growth is slower in services than in tradable industries. 

These empirical observations discipline our modeling choices. We model the spatial distribution of economic activity in a country. Space is represented by a line. As an equilibrium outcome, a section of this line is populated by residents. Services are produced in a section directly adjacent to the residential area, but manufactured products are produced farther away, where rents are lower. Because land complements other consumption goods in final utility, the demand for residential land increases with development. This crowds out other users of land, especially services. That is what leads to an increase in service prices.

An implication of the model is that the relationship between income and the price of non-tradables is stronger among highly urbanized countries than among rural countries. This is because land available for non-tradable production is more scarce in urbanized countries and the increased demand for residential land puts more pressure on rents in these countries. Empirically, this relationship is strongly borne out in the data. As we later document in Table \ref{tab:BS} for a cross section of 186 countries in 2000, the positive correlation between income and the price level is very strong among urban countries, while there is virtually no such correlation among rural countries. This result is robust to alternative definitions or urbanization (percentage of the population living in cities vs population density).

Note  that the above effect is distinct from the direct effect of development on urbanization. Richer countries tend to be more urban; more urban countries, in turn, tend to have higher prices. Even though this effect is also fully consistent with our model, we would like to test the robustness of the \emph{interaction} effect of urbanization and development. In future work, we plan to instrument urbanization rates with the geographic features within the country, such as elevation, the slope of the terrain, distance to coastline and rivers, and latitude.

We also emphasize that our results do not overturn previous explanations proposed in international macroeconomics or the regional economics literature. Instead, we view them as complementing both types of explanations and potentially bridging the gap between the two. The non-reproducible nature of residential land, together with the optimal location of industries, provides a microfoundation for the {assumption} made by Balassa and Samuelson that non-tradable industries face slower productivity growth. We believe this microfoundation is important because models that assume exogenous differences in productivity growth have no predictions about when and why the Balassa--Samuelson effect is stronger in one country than in another. Our model links the strength of the Balassa--Samuelson effect to the regional patterns of countries.

Section \ref{model} describes the model of housing, development, and relative prices. Although the necessary steps are described in some detail, formal proofs are omitted in this draft. Section \ref{empirics} provides empirical evidence on both the key cross-country implication, as well as some suggestive evidence on the mechanism of the model.

\section{A model of housing, development, and relative prices}\label{model}
We wish to study how the relative price of industries is affected by their location and how industry location, in turn, is determined by development. To model the spatial structure of the economy, we use a monocentric city model.

We model each country as an interval on the real line. There is a central business district (CBD), which is a point on the line. Businesses and residences can choose their location freely within the interval. Locations will be indexed by their distance to the CBD, denoted by $z$. The CBD serves as a marketplace: all goods and services are exchanged there.\footnote{It is straightforward to extend the model to multiple cities in each country as long as these cities do not overlap.}

Households own all the land in the country, and there are perfect rental markets for land. Households also earn income from wages.

\subsection{Technology}
There are two produced goods, we call them ``manufactured goods'' and ``services.'' Both goods are produced using land and labor and they are costly to transport to the CBD. Manufactured goods are assumed to have lower transport costs than services.

Both sectors use Cobb--Douglas technology with the same land share, $\beta$.\footnote{We assume identical land shares to highlight that our results do not hinge on factor share differences. Later, in our empirical application, we allow for different factor shares. The key results would also go through with any twice differentiable constant-returns-to-scale production.}
\begin{align*}
m&=A_ml_m^\beta n_m^{1-\beta},\\
s&=A_sl_s^\beta n_s^{1-\beta},
\end{align*}
where $l_i$ is the amount of land, and $n_i$ is the number of workers allocated to sector $i$, and $A_i$ is total factor productivity in sector $i$. Technical progress will be captured as an increase in $A_i$.

Housing services are produced using land only,
\[
 h=A_hl_h,
\]
where $l_h$ is land allocated to housing, and $A_h$ is the productivity of residential land (which can be increased, for example, by building taller and better structures).

\subsection{Tastes}
Consumers consume manufactured goods ($m$), services ($s$) and housing ($h$). They commute to the CBD to work and to buy the two goods.

There is a continuum of identical consumers with mass $N$. Utility is characterized by a homothetic utility function,
\[
u(m,s,h),
\]
so that indirect utility can be written as
\[
u[I, p_m,p_s,p_h(z)] = \frac{I}{P[p_m,p_s,p_h(z)]}.
\]
Here $p_m$ is the price of manufacturing, $p_s$ is the price of services, and $p_h(z)$ is the rental price of housing in location $z$.

We assume a nested utility structure so that we can think of the bundle of consumption goods as one good,
    \[
P[p_m,p_s,p_h(z)] = P[\Phi(p_m,p_s),p_h],
    \]
where $\Phi$ is a price aggregator function, and $P$ and $\Phi$ are both homogeneous of degree one.

\subsection{Transport costs}
All transport and commuting costs are of the iceberg nature. When a unit of good $i$ is shipped $z$ miles, only $D(\tau_i z)$ fraction of it remains. The function $D()$ is between 0 and 1, continuous and is strictly decreasing in $z$.

The profit of a firm in industry $i$, $z$ miles from the center is
\[
\pi_i(z,l,n) = p_iD(\tau_i z)A_il_i(z)^\beta n_i(z)^{1-\beta} - wn_i(z) - r(z)l_i(z),
\]
where $w$ is the wage rate, and $r(z)$ is the land rent in location $z$.\footnote{Note that the wage does not depend on the location. We assume that all employees are hired in the CBD.} Profits are revenues net of transport cost, minus all production costs.

To maintain the convenient iceberg nature of transport costs for residential location, we assume that commuting $z$ miles costs a $1-D(\tau_h z)$ fraction of the consumption bundle (including manufacturing, services and housing). We can think of this as time lost in commuting from enjoying the consumption goods. Then the budget constraint of a household at location $z$ is
\[
p_m m(z) + p_s s(z) + p_h(z) h(z) \le (w+T)D(\tau_h z).
\]
Expenditure on the manufactured good, on services and housing has to be less than the total income of the household net of commuting costs. Total income is made of wages $w$ and rental income $T$. Importantly, neither source of income depends on the choice of residential location. Rental income is derived from the plot of land the household \emph{owns}, not from where it \emph{resides}. Perfect rental markets ensure that the ownership of the land is irrelevant for location decisions.

We assume that $\tau_h>\tau_s>\tau_m$ so that commuting is the costliest of all transportation activities. This condition is sufficient to ensure the spatial structure we characterize below: residents live closest to the CBD, followed by service and by manufacturing establishments. We believe this assumption captures key aspects of modern urban structure. Alternatively, we could introduce non-pecuniary externalities that would make residents prefer to live near other residents and services establishments.\footnote{As a thought experiment, the reader can rank the following three neighborhoods for choice of residence: (i) a residential area, (ii) a neighborhood with high concentration of restaurants, theaters and other service establishments, (iii) a manufacturing area.} For historical comparisons, we also wish to vary the relative transport costs.


\subsection{Equilibrium}
\begin{definition}
 An \emph{equilibrium} in this economy is a collection of land, $\{l_i(z)\}$, and labor allocations, $\{n_i(z)\}$ and goods production, $\{y_i(z)\}$, in each industry $i$ at each location $z$; a collection of consumptions, $(m,s,\{h(z)\})$; and a list of good and factor prices, $(p_m,p_s,\{p_h(z)\},w,\{r(z)\})$ such that
\begin{enumerate}
 \item firms maximize profits (and hence choose their location optimally),
 \item households maximize utility (and hence choose their residence optimally),
 \item manufacturing and service markets clear at the CBD,
 \item the labor market clears at the CBD,
 \item land and housing markets clear at each location.
\end{enumerate}
\end{definition}
We first characterize the profit maximization problem. The first-order condition for profit maximum is a non-positive profit condition,
\[
p_i D(\tau_i z) \le \frac{r(z)^\beta w^{1-\beta}}{A_i\beta^\beta (1-\beta)^{1-\beta}},
\]
with equality whenever there is positive production. The left-hand side is the marginal revenue from selling one more unit of good $i$ in a market $z$ miles away. The right-hand side is the minimum unit costs of producing good $i$ at location $z$.

In what follows, we normalize the wage rate to 1 without loss of generality.

The zero-profit condition pins down industry $i$'s bid rent function,
\begin{equation}\label{eq:bidrent:industry}
R_i(z,p_i) = \beta (1-\beta)^{1/\beta-1} p_i^{1/\beta} A_i^{1/\beta} D(\tau_i z)^{1/\beta}.
\end{equation}
This is the maximum rent the industry is willing to pay at location $z$. It is decreasing in $z$. Note that, all else equal, the bid rent function is steeper for services because $\tau_s>\tau_m$. Figure \ref{fig:bid-rent-2} plots two bid-rent curves with linear transport costs and $\beta=1$.

\dofigure{bid-rent-2}{Two bid rent curves}

\bigskip

Consumers' choice can be broken down to two steps. The first step is the allocation of consumption expenditure at a given location to manufacturing, services, and housing. Once that decision is made optimally, the resulting utility at location $z$ will be characterized by the indirect utility function of income and prices.

Free mobility ensures that, in optimum, all inhabited locations yield the same utility. Denote this level of utility by $u$. (This is an endogenous variable that we characterize later.)

Given prices and incomes, the utility achieved at location $z$ is
    \[
    u = \frac{D(\tau_h z) I}{P[\Phi(p_m,p_s),r(z)/A_h]}.
    \]
Here we substituted in the price of housing as $p_h(z) = r(z)/A_h$.

This equation defines a bid rent function for the consumer
\begin{equation}\label{eq:bidrent:hh}
R_h(z,I,u,p_m,p_s) = A_h\Phi(p_m,p_s)P_2^{-1}\left[\frac{D(\tau_h z) I/u}{u\Phi(p_m,p_s)}\right].
\end{equation}
This is the rent that yields utility $u$ for a consumer with income $I$ at location $z$ when the two other good prices are $p_m$ and $p_s$. Since the price index $P$ is convex in its arguments, $P_2$ is monotonic. Hence the bid rent function is decreasing in $z$. Locations closer to the CBD reduce commuting costs and hence command higher rents.

Suppose, for example, that utility is Cobb--Douglas, $u(m,s,h) = m^\alpha s^\gamma h^\delta$. In that case, the bid rent curve simplifies to
\[
R_h(z,I,u,p_m,p_s) = A_h\left[\frac{D(\tau_h z) I}{up_m^{\alpha}p_s^{\gamma}}\right]^{1/\delta}.
\]

\begin{proposition}\label{prop:spatial}
An equilibrium exists, is unique, and exhibits the following spatial structure. Residents live closest to the center, followed by service establishments, and then by manufacturing establishment.
\end{proposition}

This spatial structure permits us to construct the equilibrium as follows. Let $z_1$ denote the boundary between the residential and the services zone, and $z_2$ the boundary between services and manufacturing. The boundary of the city is given by $z_3: D(\tau_m z_3)=0$.\footnote{Note that $z_3$ may be infinite.} These cutoffs pin down the amount of land dedicated to each industry.

Because land rents are a constant fraction of revenue, we can use \eqref{eq:bidrent:industry} to express output per given amount of land as
\[
y_i(z) = (1-\beta)^{1/\beta-1} p_i^{1/\beta-1} A_i^{1/\beta} D(\tau_i z)^{1/\beta}.
\]
The supplies in the two industries are
\begin{align*}
s &= (1-\beta)^{1/\beta-1} p_s^{1/\beta-1} A_s^{1/\beta} \int_{z_1}^{z_2}D(\tau_s z) dz, \\
m&= (1-\beta)^{1/\beta-1} p_m^{1/\beta-1} A_m^{1/\beta} \int_{z_2}^{z_3}D(\tau_m z) dz.
\end{align*}
The next key equation comes from location arbitrage at the manufacturing--service boundary, $z_2$. At this point, $R_s(z_2)=R_m(z_2)$, which follows from the continuity of $D(z)$. Hence a boundary $z_2$ pins down the relative price of services,
\begin{equation}\label{eq:arbitrage}
\frac{p_s}{p_m} = \left[\frac{D(\tau_m z_2)}{D(\tau_s z_2)}\right]^{1/\beta}.
\end{equation}
Note that because $\tau_s>\tau_m$ and $D(z)$ is decreasing in $z$, the farther out the boundary is, the higher this relative price. Intuitively, it is equally profitable to produce services and the manufactured good at the boundary. The farther out the boundary, the higher the transport cost differential between the two industries. This has to be compensated by higher service prices.

The relative price of services, in turn, determines relative demand. (Because of the nested utility structure, housing does not affect the relative demand of services and manufacturing.)

We can construct the supply and demand of housing similarly. We then need to find $z_1$ and $z_2$ such that markets clear.

\subsection{Productivity growth}
We conduct the following comparative statics. We increase $A_m$ and $A_s$ proportionally (so that productivity growth is neutral), and ask what happens to industry location ($z_1$, $z_2$) and relative prices.

The following proposition states that without any technological differences across the sectors, their relative price is still affected by the location choice of the sectors.
\begin{proposition}[Balassa--Samuelson and the sprawl]\label{prop:sprawl}
 Assume that $A_m/A_s$ is constant. The the relative price of services increases with development if and only if residential land grows with development.
\end{proposition}
\begin{proof}
We can determine the relative price of services from rent arbitrage at boundary $z_2$, equation \eqref{eq:arbitrage}, which is clearly increasing in $z_2$. The relative demand for services is a decreasing function of the relative price,
    \[
    \frac{s}{m} = \phi\left(\frac{p_s}{p_m}\right) = \phi\left[\frac{D(\tau_m z_2)}{D(\tau_s z_2)}\right],
    \]
where $\phi$ denotes $\Phi_2/\Phi_1$. The relative supply of services is increasing in $z_2$ for any given $z_1$, since as $z_2$ increases, more land gets allocated to services and less to manufacturing. Hence, for a given $z_1$, there is a unique $z_2$ that equates relative demand and supply. This $z_2$ is increasing in $z_1$. Substituting $z_2$ in the relative price equation above, we get the result.
\end{proof}

The intuition behind this result is that as residential land grows, both sectors have to locate farther away from the CBD. However, this affects the service sector, which has a higher transport cost, disproportionately more, and it becomes relatively more expensive.

The next question is whether residential land grows with development. The next section shows some empirical evidence that this has indeed been the case in the past decades in the U.S. In the model, we need to impose certain condition to ensure this. The following proposition introduces some necessary conditions.

\begin{proposition}[Balanced growth]
Productivity growth does not change the relative price of services if either
      (i) housing productivity grows at the same rate,
      or (ii) demand for housing is Cobb--Douglas.
\end{proposition}
If either condition (i) or condition (ii) holds, then the amount of land dedicated to housing is constant with development, which, by Proposition \ref{prop:sprawl}, implies that the relative price of services is constant.

\subsection{A special case}
To explore the properties of the model further, we make the following additional assumptions.
There is no technical progress in housing, $A_h$ is constant. Utility is Cobb--Douglas in goods, Leontief in housing,
\[
u(m,s,h) = \min\{m^\gamma s^{1-\gamma} ,h/H\}.
\]
Transport costs are exponential,\footnote{This corresponds to a constant hazard rate. If, for example, during each mile transported there is a 1\% chance that the goods get stolen, the iceberg shipping cost is $\exp(-0.01 z)$.}
\[
D(\tau z) = \exp(-\tau z).
\]
These assumptions lead to a closed-form solution.

The strength of the Balassa--Samuelson effect can be obtained by (log-)differentiating the relative price of services with respect to productivity:
\begin{equation}\label{eq:bscs}
\frac{d\ln (p_s/p_m)}{d\ln A} = \frac{(\tau_s-\tau_m)z_1}{1+\Bar\tau z_1/\beta},
\end{equation}
where $\Bar\tau$ is the consumption-share-weighted average of manufacturing and services transport costs.

The Balassa--Samuelson effect is stronger if (i) trade cost differential between the two industries is large, (ii) land occupied by residential housing is large, and (iii) the land share in production is large.

The model yields the following testable predictions. 
\begin{proposition}\label{prop:compstat}
As productivity increases,
\begin{enumerate}
    \item residential land grows,
    \item home prices increase,
    \item the rent gradient becomes steeper,
	\item tradable industries move away from center,
    \item services become more expensive,
	\item labor productivity increases faster in manufacturing.
\end{enumerate}
\end{proposition}
\begin{proposition}
All of the effects discussed in proposition \ref{prop:compstat} are stronger (in absolute value) in countries with higher population per CBD (``urban'' countries).
\end{proposition}
%% keep tau so that we can do comparative statics

%% i think if taum decreases, the BS effect becomes stronger, but will have to verify before writing up


\section{Empirical results}\label{empirics}
In this section, we present some suggestive time-series and cross-section evidence based on high-quality U.S.~data supporting the main predictions of our model formulated in Proposition \ref{prop:compstat}.

\subsection{Land and development}
In this section, we first present U.S.~time-series evidence showing that over the past decades both residential land area and residental land prices have increased. Because both quantities and prices have gone up, this is consitent with an increase in (per capita) housing demand. Then, using mainly cross-sectional evidence, we show that income is a major factor influencing residential demand as implied by Proposition \ref{prop:compstat}. The trend increases in housing demand can then be thought of as the by-product of U.S.~growth.

Looking at U.S.~time-series data, Overman, Puga and Turner, 2006 find that the per capita residential land use between 1976 and 1992 \emph{increased by 25.4\%}, based on the highest available resolution data\footnote{The 1992 data is based on 8.7 million 30 x 30 meters cells.} of the US land use based on aerial and satellite photographs. Davis and Heathcote, 2007 find that the implied real price of residential land\footnote{To get this data, the authors use
data on construction costs and decompose the value of an
average home to replacement value and land value.} has increased substantially during the period (between 1950 and 2000 the residential land prices increased nine-fold) -- clearly showing increasing residential demand for land over the period. Figure \ref{fig:dh-shares} also shows that the share of land in the value of the aggregate housing stock increased from 10.4\% to 36.4\%.

\dofigure{dh-shares}{Land prices and the share of land in home value\newline
\small Source: Davis and Heathcote (2007).}

%% the rent gradient may go to the relative price discussion

Burchfield, Overman, Puga and Turner, 2006 document that US commercial land is, on average, substantially more scattered than residential land, and the probability that a commercial land is surrounded by undeveloped land is much higher in 1992 than it was in 1976.\footnote{A 30 x 30 meter cell is categorized as developed if more than 30\% of its area is covered by artificial materials.} It implies that an average commercial land user does not need to compete directly for scarce land close to residents. This hypothesis is further supported by the relative price developments of residential and farm land derived by Davis and Heathcote, 2007 and shown by figure \ref{fig:dh-prices}. The graph reveals some short term relationship between the two prices, but also shows increasing rent gradient (gap between residential and land prices) in the long term providing saving opportunities for sectors willing to locate further from the consumers.

\dofigure{dh-prices}{Real residential and farm land prices (log scale)\newline
\small Source: Davis and Heathcote (2007).}

Turning to evidence supporting that income growth is mostly responsible for the increased demand for residential land, Davis and Heathcote, 2007 present some interesting time-series results. They regress implied real land prices (in logs) on per capita real disposable personal income (in logs), nominal interest rate and inflation rate using a quarterly dataset between 1975:4 and 2005:4. They find significant coefficients with an income elasticity of 2.6. \footnote{Note, that using the cross-sectional dataset of Davis and Palumbo (2008) with implied residential land prices for 46 large U.S.~metropolitan areas, we find very similar results (see table \ref{tab:land}, column 3).}

Turning to cross-sectional evidence, figure \ref{fig:sc_davis} plots the (log) price of land against per capita income for U.S.~cities. It strongly supports our conjecture that higher income areas face more appreciated land prices.

\dofigure{sc_davis}{Land prices and income across large U.S.~cities}

More formally, the first three columns of Table \ref{tab:land} show the estimated effects of income on land prices,  replacement values (constr) and home values in U.S.~metropolitan areas. We also control for the size of the MSA. The regressions show that the regional income influences all the values, but it has the largest effect on the land prices with an amplification factor of 2.8. The last column of table \ref{tab:land} regresses the average per capita number of rooms on income and population using county level data published by the County Business Patterns. It formally supports the prediction that higher income does indeed lead to higher housing and thus land demand as implied by Proposition 4.

\begin{table}[h!]
\center \caption{Land prices, demand for housing and income (Robust standard errors are in parentheses)} \label{tab:land}
\begin{tabular}{c|ccc|c}
  \hline\hline
  Explanatory    & \multicolumn{4}{c}{Dependent variable} \\
  variables      & Land price  & Constr. price & Home price&Rooms\\
                 &  (log)      &    (log)      & (log)     & per capita \\ \hline
  log(income)    & \textbf{2.77}   & \textbf{0.38}  & \textbf{1.66} & \textbf{0.26}\\
                 & (0.67)          & (0.11)         &  (0.38) & (0.08)\\
  log(population)& 0.13            &  0.02          &  0.05   & \textbf{-0.07}\\
                 & (0.18)          & (0.03)         &  0.10   & (0.01)\\
\hline
  $R^2$          & 0.42            & 0.25           &   0.52  & 0.26\\
  No. of obs.    & 46              & 46             &   46    & 3219 \\ \hline\hline
\end{tabular}
\end{table}

\subsection{Industry location}
In this section, we first overview U.S.~cross-sectional evidence supporting the spatial equilibrium predicted by Proposition \ref{prop:spatial} that non-tradable goods (services) are produced closer to the consumers than tradables (manufacturing).\footnote{Agriculture and mining, as a result of their relatively low trade costs and high land share, are naturally located in the least densely populated counties (see Holmes and Stevens, 2003, Table 10.)} Second, we present post-war U.S.~time-series data showing that this relative spatial position of the two sectors were reinforced by development as stated in Proposition 4.

A way to measure the co-location of population and sectors is to calculate the sector's locational quotient ($q_i$), which is measured as the ratio of an area's ($i$) share in the U.S.~sectoral employment ($s_i$) and its share in the U.S.~overall employment ($x_i$, $q_i=s_i/x_i$). If $q_i$ is below 1, the area is underrepresented in that sector, while a locational quotient substantially higher than 1 means the area is specialized in the certain sector. Figure \ref{fig:LQ_scatter} plots the locational quotient in manufacturing for 3219 U.S.~counties obtained from the County Business Patters dataset of the Census Bureau\footnote{The advantage of using counties is that they cover the whole country and capture population clustering better than state level data. A potential problem arises if a large number of the employees live and work in different counties. Metropolitan statistical areas, that are defined to take commuting habits into consideration, are a potential alternative. However, naturally, they do not cover the whole country. Our results are robust to using MSAs.} as a function of the population density of the county. The figure strongly supports our claim that manufacturing is underrepresented in the dense counties, and implies the opposite relationship (not shown) for services.

\dofigure{LQ_scatter}{Locational quotient and population density\newline
 \small Source: 2000 U.S.~Census}

To examine the effects more formally, we regressed the county level relative non-tradable/tradable employment on different measures of urbanization and also included the (log) per capita median income.

\begin{table}[h!]
\center \caption{Urbanization and relative NT/T employment (Robust
standard errors are in parentheses)}
\begin{tabular}{c|cc}
  \hline\hline
  Explanatory & \multicolumn{2}{c}{Dependent variable} \\
  variables & \multicolumn{2}{c}{$\log (N_{NT}/N_T)$} \\ \hline
  urban         & \textbf{0.62} &  \\
                & (0.02)        &  \\
  log(density)  &               & \textbf{0.07} \\
                &               & (0.01) \\
  log(income)   & \textbf{0.11} & \textbf{0.18} \\
                & (0.02)        & (0.03) \\
  constant      & -0.53         & \textbf{-1.18} \\
                & (0.28)        & (0.30) \\ \hline
  $R^2$         & 0.20          & 0.10 \\
  No. of obs.   & 3218          & 3218 \\ \hline\hline
\end{tabular}
\end{table}

The results show that even after controlling for the median income,
the level of urbanization -- measured in the proportion of urban
population or population density -- significantly increases the
proportion of employees working in non-tradable sectors. This is consistent with the spatial equilibrium in our model, where services are located closer to residents than manufacturing is.

To see how development affects the location of industries, we turn to the time-series evidence. The post-war deconcentration of manufacturing has been widely documented in the literature. Holmes and Stevens, 2004 document that the spatial distribution of manufacturing has become much more even between 1947 and 1999, and -- in line our previously presented results -- became substantially underrepresented in areas with high population density. Desmet and Fafchamps, 2006, similarly, find that manufacturing has become more deconcentrated between 1970 and 2000. Because of the very high spatial correlation of population and employment, it is not surprising that the same studies have also found that the spatial distribution of services developed in line with those of the population. 

%% kitoroltem a labjegyzetet - tul bonyolult volt nekem

The time series development of the relative manufacturing employment is presented in Figure \ref{fig:deurbanization}, which plots the share of the 100 most densely populated counties in total employment and in manufacturing between 1986 and 2005. The figure shows gradual deconcentration both in terms of employment and manufacturing, but a faster decrease of the relative share of manufacturing. This implies that relative to services, manufacturing has moved farther away from residents; in line with the predictions of our model.

\dofigure{deurbanization}{Share of 100 most densely populated counties\newline
\small Source: County Business Patterns}

\subsection{Urbanization and prices}
In this section, we present some international and U.S.~evidence suggesting that the level of urbanization influences the relationship between income and price level developments in line with the predictions of our model.

Our model suggests a spatial explanation for the basic assumption of Balassa-Samuelson model with an important testable aggregate implication, namely, that the level of urbanization (population density) in a country strongly influences the relationship between aggregate and relative labor productivity, and thereby the development of the relative price ($P_{NT}/P_T$) and the aggregate price level. Using the proportion of urban population to divide the 2000 sample of the Penn World Table equally to more rural and more urbanized countries, Figure \ref{fig:sc_penn} shows the cross-country relationships of the (log) price levels and the (log) per capita GDP for both groups. The graphs shows, in line with Proposition 5,
that the relationship between income and the price level is stronger
among more urbanized countries. It also shows that the comparison is
valid as there are reasonable variation in per capita GDP for both
groups, even if there are clearly more high-income countries among
the more urbanized ones.

\dofigure{sc_penn}{Development, urbanization, and the price level}

To approach the question more formally, we use two measures for the urbanization level of a country: the proportion of urban population and the population density, both available for a large number of countries from the World Development Indicators database.\footnote{As our main question is the difference in the `closeness of residents,' both measures are imperfect, but we consider the proportion of urban population a better measure, as it can be expected to capture the clustering of the population better than just the average population density in a country. As the the definition of urban population is different in every country, however, the cross-country results should be treated with some caution.}

We estimated cross-country regressions in the following form
\begin{equation}
\log{P}=\alpha_1+\alpha_2\log Y+\alpha_3Z+\alpha_4(Z-\bar{Z})(\log Y-\overline{\log Y})+\varepsilon,
\end{equation}
where $P$ refers to the various price measures, Y is the measure of per capita GDP, and $Z$-s are the different urbanization measures. Our main
interest is in the (demeaned) cross-term, as it can be expected to
capture how the level of urbanization influences the relationship
between the output and the price measures.

Our main price measure is the comparable price level available for
over 180 countries from the Penn World Table.\footnote{The sample
extends the 115 benchmark countries, that directly participated the
1996 round of the International Comparison Project(ICP), by using
somewhat less reliable price data and other observable variables.
Our results are robust to restricting the sample to the benchmark
countries.} To check the robustness of the results, we also
calculated a non-tradable/tradable relative price index
($P_{NT}/P_{T}$) using the basic heading level price data for
consumption for the 1996 benchmark ICP countries. This relative
price is more in line with the main Balassa-Samuelson proposition
implying that the main reason of the price level difference is the
relatively more expensive non-tradable prices. The measure, however,
can be expected to be less reliable than the aggregate price level
as the non-tradable prices, like services, tend to be less
comparable internationally than tradable prices.

\begin{table}[h!]
\caption{Balassa-Samuelson regressions with urbanization (Robust
standard errors are in parentheses)} \center \label{tab:BS}
\begin{tabular}{c|cc|cc}
  \hline\hline
  Explanatory & \multicolumn{4}{c}{Dependent variables} \\
  variables &\multicolumn{2}{c}{$\log P$} & \multicolumn{2}{c}{$\log(P_{NT}/P_T)$} \\ \hline
  $\log Y$ & \textbf{0.25} & \textbf{0.34} & \textbf{0.28}   & \textbf{0.33} \\
           & (0.05)        & (0.03)        & (0.07)          & (0.05)        \\
  urban    & \textbf{0.61} &               & 0.12            &               \\
           & (0.21)        &               & (0.34)          &               \\
  urbanX   & \textbf{0.38} &               & \textbf{0.43}   &              \\
           & (0.11)        &               & (0.19)          &              \\
  log(density) &           & -0.02         &                 & -0.02        \\
             &             & (0.02)        &                 &  0.04        \\
  densityX &               & \textbf{0.05} &                  & 0.01         \\
           &               & (0.02)        &                 & 0.02         \\
  constant & \textbf{1.13} & \textbf{0.82} & \textbf{-2.96}  & \textbf{-3.13}\\
           & (0.31)        & (0.26)        & 0.50            & (0.41)        \\ \hline
  No. of obs. & 186        & 183           & 113             & 113          \\
  $R^2$    & 0.50          & 0.46          & 0.38            & 0.34  \\
  \hline\hline
\end{tabular}
\end{table}

The regressions reported in Table \ref{tab:BS} are consistent with our predictions. The first two columns use the (log) price levels as
dependent variables, while the last two uses the (log) relative
non-tradable/tradable prices. For both dependent variables, we
present two regressions with the two urbanization measures. 'Urban'
is the proportion of urban population, and 'density' is the
population density; and 'urbanX' and 'densityX' are the demeaned
output-urbanization cross terms.

In line with well known previous results, higher per capita output
increases both the price level and the relative prices across
countries, but this relationship is significantly influenced by the
level of urbanization, even after controlling for direct price
effects of the urbanization. Looking at the first regression, we
have found that 1 percent higher per capita output increases the price
level by 0.25 percent for average level of urbanization (54 percent), and 1 percent higher urbanization has a further 0.61 percent direct effect. The cross effect, however, implies that the effect of output on the prices would be close to 0 under the (extreme) case of 0 urbanization
($0.25-0.54\cdot0.38=0.05$) and would be over 0.4 in case of full
urbanization. The effects using the relative prices are similar
supporting the conjecture that the price differences are the result
of different non-tradable prices. Using the population density as
explanatory variable still supports the effect using the price
level, though it turns out to be insignificant for our sample in
explaining the cross country relative price differences.

Turning to sector-level U.S.~data, table \ref{tab:infl} gives some suggestive evidence on the relationship between proximity to the population (measured here as the employment share in the 100 most dense counties) and the long term inflation rates (average value-added inflation rates between 1947-2006) across the three major sectors. These results also support the pricing predictions of our model formulated in Proposition 4.

\begin{table}[h!]
\center
\caption{Proximity to population and inflation} \label{tab:infl}
\begin{tabular}{l|cc}
\hline\hline
& Employment share & Inflation \\
& of the most dense counties & (1947-2006, va)\\ \hline
GDP & 31.5\% & 3.4\% \\ \hline
Agriculture & 3.2\% & 0.5\% \\
Manufacturing & 23.3\% & 2.1\% \\
Services & 33.4\% & 3.5\% \\ \hline\hline
\end{tabular}
\end{table}


\section{Conclusion}
[TO BE WRITTEN]

\begin{thebibliography}{99}
\bibitem{Alessandria07} Alessandria, G., 2007, ``Pricing-to-Market and the Failure of Absolute PPP,'' \emph{Federal Reserve Bank of Philadelphia Working Paper}, 07-29.
\bibitem{Balassa64} Balassa, B., 1964, ``The Purchasing-Power Parity Doctrine: A Reappraisal,'' \emph{Journal of Political Economy}, 584-596.
\bibitem{Burchfield06} Burchfield, M., Overman, H.G., Puga, D. and Turner, M.A., 2006, ``Causes of Sprawl: A Portrait from Space," \emph{Quarterly Journal of Economics}, 587-633.
\bibitem{Canzoneri99} Canzoneri, M.B., Cumby, R.E., Diba, B., 1999, ``Relative Labor Productivity and the Real Exchange Rate in the Long Run: Evidence for a Panel of OECD Countries,'' \emph{Journal of International Economics}, 245-266.
\bibitem{Chaterjee01} Chatterjee, S. and Carlino, G. A., 2001, ``Aggregate Metropolitan Employment Growth and the Deconcentration of Metropolitan Employment," \emph{Journal of Monetary Economics}, 549-583.
\bibitem{Chinn00} Chinn, M.D., 2000, ``The Usual Suspects? Productivity and Demand Shocks and Asia-Pacific Real Exchange Rates,'' \emph{Review of International Economics}, 20-43.
\bibitem{Davis07} Davis, M. A. and Heathcote, J., 2007, ``The Price and Quantity of Residential Land in the United States,'' \emph{Journal of Monetary Economics} 54:2595-2620.
\bibitem{Davis07} Davis, M. A. and Palumbo, M. G., 2008, ``The Price
of Residential Land in Large U.S.~Cities,'' \emph{Journal of Urban Economics}, 63:352--384.
\bibitem{} Davis, M. A. and Ortalo-Magne, F., 2007, ``Household Expenditures, Wages, Rents,'' working paper.
\bibitem{Desmet06} Desmet, K. and Fafchamps, M., 2006, ``Employment Concentration across U.S.~Counties,'' \emph{Regional Science and Urban Economics}, 482-509.
\bibitem{Egert03} Egert, B., Drine, I., Lommatzsch, K. and Rault, Ch., 2003, ``The Balassa-Samuelson Effect in Central and Eastern Europe: Myth or Reality?,'' \emph{Journal of Comparative Economics}, 552-572.
\bibitem{} Halpern, L. and Wyplosz, Ch., 2001, ``Economic Transformation and Real Exchange Rates in the 2000s: The Balassa-Samuelson Connection,'' \emph{United Nations Economic Commission for Europe Discussion Paper}, 2001.1
\bibitem{Heston06} Heston, A., R. Summers and B. Aten, 2006, Penn World Table Version 6.2, \emph{Center for International Comparisons of Production, Income and Prices at the University of
Pennsylvania}
\bibitem{Holmes03} Holmes, Th.J. and Stevens, J.J., 2004, ``Spatial Distribution of Economic Activities in North America,'' in: J. V. Henderson and J. F. Thisse (ed.) \emph{Handbook of Regional and Urban Economics}, chapter 63, 2797-2843
\bibitem{Klenow07} Hsieh, C.T and Klenow, P.J, ``Relative Prices and
Relative Prosperity,'' \emph{The American Economic Review}, 562-585.
\bibitem{Obstfeld99} Obstfeld, M. and Rogoff, K., 1996, \emph{Foundations of International Macroeconomics}, The MIT Press, Cambridge MA
\bibitem{Overman06} Overman, H.G., Puga, D., Turner, M.A., 2006, ``Decomposing the Growth in Residential Land in the United States,'' working paper.
\bibitem{Samuelson64} Samuelson, P. A. , 1964, ``Theoretical Notes on Trade Problems,'' \emph{The Review of Economics and Statistics}, 145-154.
\bibitem{} Tang, X., 2007, ``The Rich Neighborhood Effect versus the Balassa-Samuelson Effect: An Income-Based Explanation of International Price Level Differences,'' working paper.
\bibitem{Valentinyi07} Valentinyi, A., Herrendorf, B., 2007, ``Measuring Factor Income Shares at the Sector Level,'' working paper.
\end{thebibliography}

\end{document}

